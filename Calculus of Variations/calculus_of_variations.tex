\documentclass{book}

\usepackage{amsmath, mathrsfs, amssymb, bbm, amsthm, enumitem, times, mathtools, mathptmx, tensor, xcolor, esint, hyperref, showlabels}
\usepackage[framemethod=tikz]{mdframed}
\usepackage[margin=4cm]{geometry}

\newcommand{\scrA}{\mathscr{A}}
\newcommand{\scrB}{\mathscr{B}}
\newcommand{\scrD}{\mathscr{D}}
\newcommand{\scrE}{\mathscr{E}}
\newcommand{\scrF}{\mathscr{F}}
\newcommand{\scrH}{\mathscr{H}}
\newcommand{\scrL}{\mathscr{L}}
\newcommand{\scrM}{\mathscr{M}} 
\newcommand{\scrN}{\mathscr{N}}
\newcommand{\bbE}{\mathbb{E}}
\newcommand{\bbN}{\mathbb{N}}
\newcommand{\bbP}{\mathbb{P}}
\newcommand{\bbR}{\mathbb{R}}
\newcommand{\bbZ}{\mathbb{Z}}
\newcommand{\bbone}{\mathbbm{1}}
\newcommand{\bfY}{\mathbf{Y}}
\newcommand{\bfGY}{\mathbf{GY}}
\renewcommand{\d}{\mathrm{d}}
\newcommand{\D}{\mathrm{D}}
\newcommand{\T}{\mathrm{T}}
\newcommand{\e}{\mathrm{e}}
\renewcommand{\i}{\mathrm{i}}
\renewcommand{\epsilon}{\varepsilon}
\renewcommand{\phi}{\varphi}

\newcommand{\abs}[1]{\left\lvert {#1} \right\rvert}
\newcommand{\fhabs}[1]{\lvert {#1} \rvert}
\newcommand{\norm}[1]{\left\lVert {#1} \right\rVert}
\newcommand{\fhnorm}[1]{\lVert {#1} \rVert}
\newcommand{\set}[1]{\left\{ {#1} \right\}}
\newcommand{\fhset}[1]{\{ {#1} \}}
\newcommand{\parens}[1]{\left( {#1} \right)}
\newcommand{\angles}[1]{\left\langle {#1} \right\rangle}
\newcommand{\fhangles}[1]{\langle {#1} \rangle}
\newcommand{\aangles}[1]{\left\llangle {#1} \right\rrangle}
\newcommand{\fhaangles}[1]{\llangle {#1} \rrangle}

\newcommand{\pdv}[2]{\frac{\partial{#1}}{\partial{#2}}}
\newcommand{\odv}[3][]{\frac{\d^{#1}#2}{\d{#3}^{#1}}}
 
\newcommand{\distributionEqual}{\overset{\scrD}{=}}
\newcommand{\iidEqual}{\overset{\mathrm{i.i.d.}}{=}}
\newcommand{\weak}{\rightharpoonup}
\newcommand{\weakstar}{\overset{\ast}{\rightharpoonup}}
\newcommand{\young}{\overset{\mathbf{Y}}{\rightarrow}}

\newcommand{\restrict}{\begin{picture}(10,8)\put(2,0){\line(0,1){7}}\put(1.8,0){\line(1,0){7}}\end{picture}}

\DeclareMathOperator{\dom}{dom}
\let\div\relax
\DeclareMathOperator{\div}{div}
\DeclareMathOperator{\sgn}{sgn}
\DeclareMathOperator{\tr}{tr}
\DeclareMathOperator{\supp}{supp}
\DeclareMathOperator{\id}{\mathrm{id}}
\DeclareMathOperator{\rank}{\mathrm{rank}}
\DeclareMathOperator{\cof}{\mathrm{cof}}

\makeatletter
\DeclareFontFamily{OMX}{MnSymbolE}{}
\DeclareSymbolFont{MnLargeSymbols}{OMX}{MnSymbolE}{m}{n}
\SetSymbolFont{MnLargeSymbols}{bold}{OMX}{MnSymbolE}{b}{n}
\DeclareFontShape{OMX}{MnSymbolE}{m}{n}{
    <-6>  MnSymbolE5
   <6-7>  MnSymbolE6
   <7-8>  MnSymbolE7
   <8-9>  MnSymbolE8
   <9-10> MnSymbolE9
  <10-12> MnSymbolE10
  <12->   MnSymbolE12
}{}
\DeclareFontShape{OMX}{MnSymbolE}{b}{n}{
    <-6>  MnSymbolE-Bold5
   <6-7>  MnSymbolE-Bold6
   <7-8>  MnSymbolE-Bold7
   <8-9>  MnSymbolE-Bold8
   <9-10> MnSymbolE-Bold9
  <10-12> MnSymbolE-Bold10
  <12->   MnSymbolE-Bold12
}{}

\let\llangle\@undefined
\let\rrangle\@undefined
\DeclareMathDelimiter{\llangle}{\mathopen}%
                     {MnLargeSymbols}{'164}{MnLargeSymbols}{'164}
\DeclareMathDelimiter{\rrangle}{\mathclose}%
                     {MnLargeSymbols}{'171}{MnLargeSymbols}{'171}
\makeatother

\def\Xint#1{\mathchoice
{\XXint\displaystyle\textstyle{#1}}%
{\XXint\textstyle\scriptstyle{#1}}%
{\XXint\scriptstyle\scriptscriptstyle{#1}}%
{\XXint\scriptscriptstyle\scriptscriptstyle{#1}}%
\!\int}
\def\XXint#1#2#3{{\setbox0=\hbox{$#1{#2#3}{\int}$ }
\vcenter{\hbox{$#2#3$ }}\kern-.6\wd0}}
\def\ddashint{\Xint=}
\def\dashint{\Xint-}

\newtheorem{theorem}{Theorem}[chapter]
\newtheorem{proposition}[theorem]{Proposition}
\newtheorem{lemma}[theorem]{Lemma}
\newtheorem{corollary}[theorem]{Corollary}

\theoremstyle{definition}
\newtheorem{example}[theorem]{Example}
\newtheorem{remark}[theorem]{Remark}

\surroundwithmdframed[outerlinewidth=0.4pt,middlelinewidth=1pt,innerlinewidth=0.4pt,middlelinecolor=white,bottomline=false,topline=false,rightline=false,innertopmargin=-9pt,innerbottommargin=-1pt]{theorem}
\surroundwithmdframed[outerlinewidth=0.4pt,bottomline=false,topline=false,rightline=false,innertopmargin=-9pt,innerbottommargin=-1pt]{lemma}
\surroundwithmdframed[outerlinewidth=0.4pt,bottomline=false,topline=false,rightline=false,innertopmargin=-9pt,innerbottommargin=-1pt]{proposition}
\surroundwithmdframed[outerlinewidth=0.4pt,bottomline=false,topline=false,rightline=false,innertopmargin=-9pt,innerbottommargin=-1pt]{corollary}
%\surroundwithmdframed[tikzsetting={draw=black,line width=1pt,dashed},bottomline=false,topline=false,rightline=false,innertopmargin=-5pt,outerlinecolor=white,middlelinecolor=white]{example}

\numberwithin{equation}{chapter}

\title{Calculus of Variations} 
\author{Billy Sumners}

\begin{document}
\maketitle 

\tableofcontents

\chapter{Introduction}

\chapter{Convexity}

\section{The Direct Method}
The following theorem will be the fundamental in our forthcoming theory. It reduces proving the existence of a minimizer of a functional to checking two properties, and even gives us the freedom to choose a nice topology.
\begin{theorem}[Direct Method]
    Let $X$ be a topological space, and let $\scrF \colon X \to \bbR \cup \set{\infty}$ be the \textit{objective functional}. Assume
    \begin{itemize}
        \item (Sequential coercivity) For all $\Lambda \in \bbR$, the set $\set{u \in X : \scrF[u] \leq \Lambda}$ is sequentially precompact in $X$.
        \item (Sequential lower semicontinuity) For all sequences $u_j \in X$ with $u_j \rightarrow u$, we have $\scrF[u] \leq \liminf_{j\rightarrow \infty} \scrF[u_j]$.
    \end{itemize}
    Then the variational problem
    \begin{equation}
        \text{minimize } \scrF[u] \text{ over } u \in X
    \end{equation}
    has a solution.
\end{theorem}
\begin{proof}
    Let $\alpha := \inf_{u \in X} \scrF[u]$. Choose a sequence $u_j \in X$ such that $\scrF[u_j] \rightarrow \alpha$. Then the sequence $\scrF[u_j]$ is bounded from above in $\bbR$, so by coercivity, we can pass to a subsequence such that $u_j$ converges to some $u \in X$. (\textit{NB} subsequences will not often be relabeled in these notes.) By lower semicontinuity, $\alpha \leq \scrF[u] \leq \liminf_{j \rightarrow \infty} \scrF[u_j] = \alpha$. So $u$ is our desired minimizer.
\end{proof}
\begin{example}
    Define $h \colon \bbR \to \bbR$ by 
    \begin{equation}
        h(t) :=
        \begin{cases}
            1-t & t < 0,    \\
            t   & t \geq 0.
        \end{cases}
    \end{equation}
    Then $h$ is coercive: for $\Lambda \geq 1$,
    \begin{equation}
        \set{t \in \bbR : h(t) \leq \Lambda} = \set{t < 0 : 1-t \leq \Lambda} \cup \set{t \geq 0 : t \leq \lambda} 
                                             = [1-\Lambda,\Lambda].
    \end{equation}
    It is easy to see check precompactness for the remaining $\Lambda \in \bbR$. Moreover, $h$ is lower semicontinuous: $h$ is continuous on $\bbR \setminus \set{0}$, and if $t_j$ is a sequence in $\bbR$ converging to $0$, then $h(0) = 0 \leq h(t_j)$ for all $j \in \bbN$. (Actually, it is clear that $0$ is the minimizer of $h$, but this example is more an illustration of the direct method.)
\end{example}
\begin{remark}
    \begin{enumerate}[label=(\arabic*)]
        \item The direct method is trivial but powerful - we will be using it constantly throughout the course and finding conditions for it to apply.
        \item Coercivity and lower semicontinuity are not necessary for the existence of a minimizer - consider the example $h$ above, but with $h(1) = 50$.
        \item When $\scrF$ is an integral functional, lower semicontinuity is related to convexity (and its relatives) of the integrand.
        \item The direct method gives us freedom to choose an appropriate topology for the problem. A weaker topology will mean lower semicontinuity is easier to prove, but coercivity is harder to prove, and a stronger topology will do precisely the opposite.
    \end{enumerate}
\end{remark}
Most commonly, we will be applying the direct method when $X$ is a reflexive Banach space with the weak topology, or an affine subspace of a reflexive Banach space with the weak topology.

\section{Functionals with Convex Integrands}
We will be considering the functional
\begin{equation}
    \scrF[u] = \int_{\Omega} f(x,\nabla u(x)) \; \d x
\end{equation}
over some appropriate Sobolev space and $\Omega$ we will discuss soon. First, a lemma:
\begin{lemma}
    Let $f \colon \Omega \times \bbR^N \to \bbR$ be a \textit{Carath\'eodory integrand}. That is,
    \begin{enumerate}[label=\rm{(\roman*)}]
        \item For all $A \in \bbR^N$, the map $f(\cdot,A) \colon \Omega \to \bbR$ is Lebesgue measurable.
        \item For $\scrL^d$-a.e. $x \in \Omega$, the map $f(x,\cdot) \colon \bbR^N \to \bbR$ is continuous.
    \end{enumerate}
    Let $V \colon \Omega \to \bbR^N$ be Borel measurable. Then the map $f(\cdot,V(\cdot)) \colon \Omega \to \bbR$ is Lebesgue measurable.
\end{lemma}
Note that we can't deduce measurability by considering $x \mapsto (x,V(x)) \mapsto f(x,V(x))$ as a composition, since $f$ is not jointly measurable in its two arguments.
\begin{proof}
    Suppose first that $V$ is simple. That is, $V = \sum_{i=1}^m a_i \bbone_{A_i}$, where each $A_i \subseteq \Omega$ is a Borel set, the $A_i$ are disjoint, and $\bigcup_{i=1}^m A_i = \Omega$. Then 
    \begin{equation}
        f(x,V(x)) = \sum_{i=1}^m f(x,a_i)\bbone_{A_i}(x),
    \end{equation}
    so given a Borel set $B \subseteq \bbR$,
    \begin{equation}
        \set{x \in \Omega : f(x,V(x)) \in B} = \bigcup_{i=1}^m \set{x \in A_i : f(x,a_i) \in B}.
    \end{equation}
    Since $f(\cdot,a_i)$ is Lebesgue measurable for all $i$, we see that each set on the right hand side is Lebesgue measurable, so the set on the left hand side is Lebesgue measurable.

    Next, suppose $V$ is arbitrary, and choose a sequence $V_i$ of simple functions such that $V_i \rightarrow V$. By continuity of $f(x,\cdot)$ for a.e. $x$, $f(\cdot,V(\cdot)) = \lim_{i \rightarrow \infty} f(\cdot, V_i(\cdot))$ is Lebesgue measurable. This property requires completeness of $(\Omega,\scrB^*(\Omega))$, where $\scrB^*(\Omega)$ denotes the Lebesgue $\sigma$-algebra.
\end{proof}

Fix $p \in [1,\infty]$. For the following theory, we define the functional $\scrF \colon \dom{\scrF} \subseteq W^{1,p}(\Omega;\bbR^m) \to \bbR \cup \set{\pm \infty}$ by 
\begin{equation}
    \scrF[u] := \int_{\Omega} f(x,\nabla u(x)) \; \d x.
\end{equation}
Here, $\Omega \subseteq \bbR^d$ is a bounded Lipschitz domain (i.e. $\Omega$ is open and $\partial\Omega$ is a Lipschitz manifold), and $f \colon \Omega \times \bbR^{m \times d} \to \bbR$ is a Carath\'eodory integrand. Furthermore, $W^{1,p}(\Omega;\bbR^n)$ is the Sobolev space consisting of functions $u \colon \Omega \to \bbR^n$, such that each component function $u^i$ is in $L^p(\Omega)$ and is weakly differentiable in every direction, and each weak derivative $\partial_j u^i$ is also in $L^p(\Omega)$. The $m \times d$ \textit{gradient matrix} $\nabla u$ has components $(\nabla u)^i_j = \partial_j u^i$.
\begin{remark} 
    The map $\scrF$ is not usually defined on the whole space $W^{1,p}(\Omega;\bbR^m)$ since the integral may not be well defined. To ensure it is well defined, we will have to impose some more conditions on the integrand $f$. Despite this, we will still typically write $\scrF \colon W^{1,p}(\Omega;\bbR^m) \to \bbR \cup \set{\pm \infty}$.
\end{remark}

{\color{red} THIS PROPOSITION NEEDS TO BE STATED IN MORE GENERALITY}
\begin{proposition} \label{prop:p-coercivityImpliesWeakCoercivity}
    Suppose $f \colon \Omega \times \bbR^{m \times d} \to \bbR$ is $p$-\textit{coercive} for some $p \in (1,\infty)$. That is, there exists $\mu > 0$ such that $\mu \abs{A}^p \leq f(x,A)$ for a.e. $x \in \Omega$ and $A \in \bbR^{m\times d}$. Then $\scrF$ is weakly coercive on the affine space $W_g^{1,p}(\Omega;\bbR^m)$, which is the space of all $u \in W^{1,p}(\Omega;\bbR^m)$ such that $u\vert_{\partial\Omega} = g$.
\end{proposition}
Here, $\cdot\vert_{\partial\Omega} \colon W^{1,p}(\Omega;\bbR^m) \to L^p(\partial\Omega;\bbR^m)$ is the usual trace operator. Its image is denoted by $W^{1-\frac{1}{p},p}(\partial\Omega;\bbR^m)$, and we must assume $g$ lives in this image.
\begin{remark}
    In the situation of proposition \ref{prop:p-coercivityImpliesWeakCoercivity}, $p$-coercivity implies the integrand $f$ is nonnegative, so our functional $\scrF$ is defined on the whole space $W^{1,p}(\Omega;\bbR^m)$.
\end{remark}
\begin{proof}
    Let $u_j$ be a sequence in $W^{1,p}_g(\Omega;\bbR^m)$ such that $\scrF[u_j] \leq \Lambda$ for some $\Lambda \in \bbR$ and all $j$. We integrate the $p$-coercivity condition $\mu\abs{\nabla u(x)}^p \leq f(x,\nabla u(x))$ over $x \in \Omega$ to find 
    \begin{equation}
        \mu\norm{\nabla u_j}^p_{W^{1,p}(\Omega;\bbR^m)} \leq \scrF[u_j] \leq \Lambda
    \end{equation}
    for all $j$. Select some $u_0 \in W^{1,p}_g(\Omega;\bbR^m)$. Since $u_j - u_0$ is in $W^{1,p}_0(\Omega;\bbR^m)$, the Friedrichs-Poincar\'e inequality implies 
    \begin{equation}
        \mu\norm{u_j - u_0}_{W^{1,p}} \leq C\mu \norm{\nabla(u_j - u_0)}_{L^p} \leq C(\Lambda + \norm{\nabla u_0}_{L^p}).
    \end{equation}
    Since $p$ lies strictly between $1$ and $\infty$, $W^{1,p}_0(\Omega;\bbR^m)$ is relexive, and so the Banach-Alaoglu theorem implies the existence of a subsequence and $u \in W^{1,p}_0(\Omega;\bbR^m)$ such that $u_j - u_0 \weak u$. But then $u_j \weak u + u_0 \in W^{1,p}_g(\Omega;\bbR^m)$, which is precisely what we wanted to prove.
\end{proof}
\begin{theorem}[Tonelli 1920, Serrin 1961] \label{thm:convexityImpliesWeakLowerSemicontinuity}
    Let $f \colon \Omega \times \bbR^{m \times d} \to [0,\infty)$ be a Carath\'e\-odory integrand such that $f(x,\cdot)$ is convex for a.e. $x \in \Omega$. Then $\scrF$ is weakly lower semicontinuous on $W^{1,p}(\Omega;\bbR^m)$ for any $p \in (1,\infty)$.
\end{theorem}
\begin{proof}
    We first prove that $\scrF$ is strongly lower semicontinuous. Let $u_j$ be a sequence in $W^{1,p}(\Omega;\bbR^m)$ converging strongly to $u$. Select a subsequence with $\nabla u_j \rightarrow \nabla u$ a.e. Then, by continuity of $f(x,\cdot)$ for a.e. $x \in \Omega$ and Fatou's lemma,
    \begin{equation}
        \scrF[u] =    \int_{\Omega} f(x,\nabla u(x)) \; \d x 
                 \leq \liminf_{j \rightarrow \infty} \int_\Omega f(x,\nabla u_j(x)) \; \d x 
                 =    \liminf_{j \rightarrow \infty} \scrF[u_j].
    \end{equation}
    This, of course, implies strong lower semicontinuity for the full sequence, since otherwise we would have a contradiction.

    Next, we strengthen(!) this to weak lower semicontinuity. Let $u_j$ be a sequence in $W^{1,p}(\Omega;\bbR^m)$ converging weakly to $u$. Write $\alpha := \liminf_{j \to \infty} \scrF[u_j]$. Select a subsequence with $\lim_{j \to \infty} \scrF[u_j] = \alpha$. By Mazur's lemma, we can select a sequence of convex combinations
    \begin{equation}
        y_j = \sum_{k=j}^{N(j)} \theta^{(j)}_k u_k; \hspace{20pt} \theta^{(j)}_k \in [0,1]; \hspace{20pt} \sum_{k=j}^{N(j)} \theta^{(j)}_k = 1,
    \end{equation}
    such that $y_j \rightarrow u$ strongly in $W^{1,p}$. Then by convexity,
    \begin{equation}
        \begin{aligned}
            \scrF[u] &\leq \liminf_{j \to \infty} \int_\Omega f(x,\nabla y_j(x)) \; \d x                                                 \\
                     &=    \liminf_{j \to \infty} \int_\Omega f\left(x, \sum_{k=j}^{N(j)} \theta^{(j)}_k \nabla u_k(x)\right) \; \d x    \\
                     &\leq \liminf_{j \to \infty} \sum_{k=j}^{N(j)} \theta^{(j)}_k \int_\Omega f(x,\nabla u_k(x)) \; \d x                \\
                     &=    \lim_{j \to \infty} \scrF[u_j]                                                                                \\
                     &=    \alpha
        \end{aligned}
    \end{equation}
    This completes the proof.
\end{proof}
Proposition \ref{prop:p-coercivityImpliesWeakCoercivity} and theorem \ref{thm:convexityImpliesWeakLowerSemicontinuity} combine to give us the following existence theorem:
\begin{theorem}
    Let $p \in (1,\infty)$, and suppose $f \colon \Omega \times \bbR^{m \times d} \to [0,\infty)$ is a $p$-coercive Carath\'eodory integrand which is convex in its second argument. Then the minimization problem 
    \begin{equation} \label{eq:integralMinimizationProblemWithoutFunctionDependence}
        \begin{dcases}
            \text{minimize } \scrF[u] = \int_\Omega f(x,\nabla u(x)) \; \d x \text{ over } u \in W^{1,p}(\Omega;\bbR^m) \\
            \textit{subject to } u\vert_{\partial\Omega} = g
        \end{dcases}
    \end{equation}
    has a solution.
\end{theorem}
If we assume \textit{strict} convexity in the second argument, then (\ref{eq:integralMinimizationProblemWithoutFunctionDependence}) has a unique solution. We state this as a theorem:
\begin{proposition} \label{prop:strictConvexityImpliesUniqueness}
    Assume the integrand $f$ in the minimization problem (\ref{eq:integralMinimizationProblemWithoutFunctionDependence}) is strictly convex in its second argument. Then any solution (if one exists) of this problem must be unique.
\end{proposition}
\begin{proof}
    Suppose $u,v \in W^{1,p}_g(\Omega;\bbR^m)$ are solutions to (\ref{eq:integralMinimizationProblemWithoutFunctionDependence}). Define $w := \frac{u+v}{2}$, which is in $W^{1,p}_g(\Omega;\bbR^m)$. Then by strict convexity,
    \begin{equation}
        \begin{aligned}
            \scrF[w] &= \int_\Omega f\left(x, \frac{\nabla u(x) + \nabla v(x)}{2}\right) \; \d x                            \\
                     &< \frac{1}{2} \int_\Omega f(x,\nabla u(x)) \; \d x + \frac{1}{2} \int_\Omega f(x,\nabla v(x)) \; \d x \\
                     &= \frac{\scrF[u] + \scrF[v]}{2}.
        \end{aligned}
    \end{equation}
    This is a contradiction, since $u$ and $v$ are minimizers.
\end{proof}
Finally, we show that in the scalar case ($m=1$) or on the line ($d=1$), weak lower semicontinuity of the integral functional $\scrF$ is actually equivalent to convexity of the integrand (with some additional assumptions).
\begin{proposition} \label{prop:wlscImpliesConvex}
    Let $p \in [1,\infty)$, and let $\scrF \colon W^{1,p}(\Omega;\bbR^m) \to \bbR$ be an integral functional with integrand $f \colon \bbR^{m \times d} \to \bbR$. If $\scrF$ is weakly lower semicontinuous, and either $m=1$ or $d=1$, then $f$ is convex.
\end{proposition}
\begin{proof}
    PROOF IS INCOMPLETE, NEED ARGUMENTS FOR WEAK CONVERGENCE

    We deal with the case $m=1$. Fix $a,b \in \bbR^d$, and $\theta \in (0,1)$. Define $v := \theta a + (1-\theta) b$ and $n := b - a$. We let 
    \begin{equation}
        \phi(t) :=
        \begin{cases}
            -(1-\theta)t & t \in [0,\theta), \\
            \theta(t-1)  & t \in [\theta,1], 
        \end{cases}
    \end{equation}
    and define $u_j \colon \Omega \to \bbR$ by 
    \begin{equation} \label{eq:definitionOfSequenceForWeakLSCImpliesConvexity}
        u_j(x) := v \cdot x + \frac{1}{j}\phi(jx \cdot n - \lfloor jx \cdot n \rfloor)
    \end{equation}
    Then 
    \begin{equation}
        \nabla u_j(x) = 
        \begin{cases}
            a & jx \cdot n - \lfloor jx \cdot n \rfloor \in [0,\theta), \\
            b & jx \cdot n - \lfloor jx \cdot n \rfloor \in [\theta,1].
        \end{cases}
    \end{equation}
    Now, $u_j \weak v \cdot x$ in $W^{1,p}$. By weak lower semicontinuity,
    \begin{equation} \begin{aligned}
        \abs{\Omega}f(\theta a + (1-\theta)b) &= \abs{\Omega}f(\nabla(v \cdot x))            \\
                                              &= \scrF[v \cdot x]                            \\
                                              &\leq \liminf_{j \to \infty} \scrF[u_j]        \\
                                              &= \abs{\Omega}(\theta f(a) + (1-\theta) f(b)),
    \end{aligned} \end{equation}
    as required.
\end{proof}
It turns out the result is false if $d \neq 1$ or $m \neq 1$, but we won't prove this here.

\section{Integrands with $u$-dependence}
Consider a functional of the form
\begin{equation} \label{eq:integralFunctionalWithFunctionDependence}
    \scrF[u] := \int_\Omega f(x,u(x),\nabla u(x)) \; \d x; \hspace{20pt} u \in W^{1,p}_g(\Omega;\bbR^m).
\end{equation}
Although the Mazur lemma used in the Tonelli-Serrin theorem \ref{thm:convexityImpliesWeakLowerSemicontinuity} does not work for this more general situation, an analog of this theorem is available in this situation:
\begin{theorem}
    Let $\Omega \subseteq \bbR^d$ be a bounded Lipschitz domain, and let $f \colon \Omega \times \bbR^m \times \bbR^{m \times d} \to \bbR$ be a Carath\'eodory integrand such that $f(x,v,\cdot)$ is convex for a.e. $x \in \Omega$ and all $v \in \bbR^m$. Then the corresponding integral functional $\scrF$, defined by (\ref{eq:integralFunctionalWithFunctionDependence}), is weakly lower semicontinuous on $W^{1,p}_g(\Omega;\bbR^m)$ for $p \in (1,\infty)$.
\end{theorem}
We won't prove it. Here's an example:
\begin{example}
    Consider the minimization problem 
    \begin{equation} \label{eq:linearizedElasticityMinimizationProblem}
        \left\{
        \begin{aligned}
            \text{minimize } \scrF[u] &= \frac{1}{2}\int_\Omega 2\mu\abs{\scrE u}^2 + \parens{\kappa - \frac{2}{3}\mu}\abs{\tr{\scrE                              u}}^2 - b \cdot u \; \d x \\
                                      &\hspace{20pt} \text{ over } u \in W^{1,2}(\Omega;\bbR^d) \\
            \text{subject to } u\vert_{\partial\Omega} &= g.
        \end{aligned}
        \right.
    \end{equation}
    For some $b \in L^2(\Omega;\bbR^d)$. Here, $\scrE u = \frac{1}{2}(\nabla u + \nabla u^T)$ is the \textit{symmetrized gradient}. For a matrix $A \in \bbR^{d \times d}$, we also define its \textit{symmetrization} to be $\scrE A := \frac{1}{2}(A + A^{\rm T})$ We would like to show that this minimization problem has a solution via the Direct Method. First, note that the integrand 
    \begin{equation}
        f(x,v,A) = \mu \abs{\scrE A}^2 + \parens{ \kappa - \frac{2}{3} \mu }\abs{\tr{\scrE A}}^2 - b(x) \cdot v
    \end{equation}
    is convex in the $A$ argument, therefore $\scrF$ is weakly lower semicontinuous.

    Secondly, we estimate
    \begin{equation} \label{eq:coercivityEstimateForLinearizedElasticity}
        \scrF[u] \geq \mu \norm{\scrE u}_{L^2}^2 - \norm{b}_{L^2} \norm{u}_{L^2}.
    \end{equation}
    In order to establish weak coercivity of $\scrF$, we need to obtain an inequality of the form $\scrF[u] \geq \widetilde{\mu}\norm{\nabla u}_{L^2}^2 - C$. To do this, we first estimate the norm of $\scrE u$ in terms of the norm of $\nabla u$. Fix $\phi \in C_c^\infty(\Omega;\bbR^d)$. We calculate 
    \begin{equation} \begin{aligned}
        2\scrE \phi : \scrE \phi - \nabla \phi : \nabla \phi &= \nabla \phi : \nabla \phi + \nabla \phi : \nabla \phi^\T - \nabla \phi : \nabla u \\
                                                             &= \nabla \phi : \nabla \phi^\T.
    \end{aligned} \end{equation}
    Integrate this over $\Omega$ to find 
    \begin{equation} \begin{aligned}
        2 \norm{\scrE \phi}_{L^2}^2 - \norm{\nabla \phi}_{L^2}^2 &= \int_\Omega \nabla \phi : \nabla \phi^\T \; \d x \\
                                                                 &= - \int_\Omega \phi \cdot \div(\nabla \phi^\T) \; \d x. \\
    \end{aligned} \end{equation}
    The components of $\div(\nabla \phi^\T)$ are given by 
    \begin{equation}
        (\div(\nabla \phi^\T))^j = \partial_i(\partial_j \phi^i) = \partial_j(\partial_i \phi^i) = \partial_j \div{\phi},
    \end{equation}
    with the Einstein summation convention in force as usual. Hence $\div(\nabla \phi^\T) = \nabla \div{\phi}$. Plugging this into our integral above, we find 
    \begin{equation} 
        - \int_\Omega \phi \cdot \div(\nabla \phi^\T) \; \d x = - \int_\Omega \phi \cdot \nabla \div{\phi} \; \d x
                                                              = \int_\Omega (\div{\phi})^2 \; \d x 
                                                              \geq 0.
    \end{equation}
    It follows that $2 \norm{\scrE \phi}_{L^2}^2 \geq \norm{\nabla \phi}_{L^2}^2$. The general case of $u \in W^{1,2}_g(\Omega;\bbR^d)$ follows from density of $C_c^\infty$ in $W^{1,2}_0$.

    For the $\norm{b}_{L^2} \norm{u}_{L^2}$ part, first pick any $u_0 \in W_g^{1,2}(\Omega;\bbR^d)$. Then by the Friedrichs-Poincar\'e inequality, there exists $C_P > 0$ independent of $u$ such that 
    \begin{equation} \begin{aligned}
        \norm{u}_{L^2} &\leq \norm{u - u_0}_{L^2} + \norm{u_0}_{L^2} \\
                       &\leq C_P \norm{\nabla u - \nabla u_0}_{L^2} + \norm{u_0}_{L^2} \\
                       &\leq C_P \norm{\nabla u} + C \norm{u_0}_{W^{1,2}},
    \end{aligned} \end{equation}
    where the constant $C$ is independent of $u$ and $u_0$. So 
    \begin{equation}
        \norm{u}_{L^2}^2 \leq 2(C_P^2 \norm{\nabla u}_{L^2}^2 + C^2 \norm{u_0}_{W^{1,2}}^2).
    \end{equation}
    By Young's inequality with $\delta > 0$, we have 
    \begin{equation} \begin{aligned}
        \norm{b}_{L^2} \norm{u}_{L^2} &\leq \frac{\delta}{2} \norm{u}_{L^2}^2 + \frac{1}{2\delta} \norm{b}_{L^2}^2 \\
                                      &\leq \frac{\delta}{2} 2(C_P^2 \norm{\nabla u}_{L^2}^2 + C^2 \norm{u_0}_{W^{1,2}}^2) + \frac{1}{2\delta} \norm{b}_{L^2}^2 \\
                                      &= \delta C_P^2 \norm{\nabla u}_{L^2}^2 + C
    \end{aligned} \end{equation}
    Choose $\delta = \frac{\mu}{4C_P^2}$. Plugging the above into (\ref{eq:coercivityEstimateForLinearizedElasticity}), we find 
    \begin{equation} \begin{aligned}
        \scrF[u] &\geq \mu \norm{ \scrE u }_{L^2}^2 - \norm{b}_{L^2} \norm{u}_{L^2} \\
                 &\geq \frac{\mu}{2} \norm{\nabla u}_{L^2}^2  - \frac{\mu}{4C_P^2} C_P^2 \norm{\nabla u}_{L^2}^2 + C \\ 
                 &= \frac{\mu}{4} \norm{\nabla u}_{L^2}^2 + C,
    \end{aligned} \end{equation}
    as required. It follows that $\scrF$ is weakly coercive, and therefore the minimization problem (\ref{eq:linearizedElasticityMinimizationProblem}) has a solution in $W_0^{1,2}(\Omega;\bbR^d)$.
\end{example}

\section{The Lavrentiev Gap Phenomenon}
Suppose $X$ is a Banach space, and $\scrF$ is a functional on $X$. If $Y$ is a dense subspace of $X$, is it necessarily true that 
\begin{equation} \label{eq:equalityOfInfimaInDenseSubspaces}
    \inf_{x \in Y} \scrF[x] = \inf_{x \in X} \scrF[x]?
\end{equation}
In the following, we will see that this isn't the case. For now, we provide a situation in which this is true.
\begin{theorem}
    Let $f \colon \Omega \times \bbR^m \times \bbR^{m \times d} \to \bbR$ be a Carath\'eodory integrand satisfying, for some $p \in [1,\infty)$, the $p$-growth condition
    \begin{equation}
        \abs{f(x,v,A)} \leq M(1 + \abs{v}^p + \abs{A}^p) \hspace{20pt} \text{for a.e. } x \in \Omega, \text{ and all } (v,A) \in \bbR^m \times \bbR^{m \times d}.
    \end{equation}
    Then the corresponding integral functional $\scrF$, defined by (\ref{eq:integralFunctionalWithFunctionDependence}), is strongly continuous on $W^{1,p}(\Omega;\bbR^m)$. Consequently, (\ref{eq:equalityOfInfimaInDenseSubspaces}) holds for $X = W^{1,p}(\Omega;\bbR^m)$ and $Y$ a dense subspace.
\end{theorem}
\begin{proof}
    Let $u_j \in W^{1,p}(\Omega;\bbR^m)$ be a sequence converging to $u$ in $W^{1,p}$. Choose an arbitrary subsequence of $u_j$, and pass to a further subsequence with $u_j \rightarrow u$ and $\nabla u_j \rightarrow \nabla u$ pointwise a.e. We have
    \begin{equation} \begin{aligned}
        \abs{ f(x,u_j(x),\nabla u_j(x)) } &\leq M(1 + \abs{u_j(x)}^p + \abs{\nabla u_j(x)}^p)  \\
                                          &\rightarrow M(1 + \abs{u(x)}^p + \abs{\nabla u(x)}^p) \hspace{10pt} \text{pointwise a.e.}
    \end{aligned} \end{equation}
    Furthermore, 
    \begin{equation} \begin{aligned}
        \int_\Omega M(1 + \abs{u_j(x)}^p + \abs{\nabla u_j(x)}^p) \; \d x &= M(\abs{\Omega} + \norm{u_j}_{W^{1,p}}^p + \norm{\nabla u_j}_{W^{1,p}}^p) \\
                                                                          &\rightarrow M(\abs{\Omega} + \norm{u}_{W^{1,p}}^p + \norm{\nabla u}_{W^{1,p}}^p) \\
                                                                          &= \int_\Omega M(1 + \abs{u(x)}^p + \abs{\nabla u(x)}^p) \; \d x.
    \end{aligned} \end{equation}
    By Pratt's lemma \ref{lem:pratt},
    \begin{equation} \label{eq:strongContinuityOfIntegralFunctionals}
        \scrF[u_j] = \int_\Omega f(x,u_j,\nabla u_j) \; \d x \rightarrow \int_\Omega f(x,u,\nabla u) \; \d x = \scrF[u].
    \end{equation}
    Since every subsequence of the original sequence $u_j$ has a further subsequence satisfying (\ref{eq:strongContinuityOfIntegralFunctionals}), we conclude this must hold for the original sequence.
\end{proof}
Next, we exhibit an example of a functional for which (\ref{eq:equalityOfInfimaInDenseSubspaces}) does not hold. Such a gap between infima is called the \textit{Lavrentiev Gap Phenomenon}.
\begin{example}[Mani\`a]
    Define 
    \begin{equation}
        \scrF[u] = \int_0^1 (u(t)^3 - t)^2 \dot{u}(t) \; \d t \hspace{15pt} u \in W_g^{1,\infty}((0,1)),
    \end{equation}
    where $g(0) = 0$ and $g(1) = 1$. We will show that 
    \begin{equation}
        \inf_{u \in W^{1,1}_g} \scrF[u] < \inf_{u \in W^{1,\infty}_g} \scrF[u].
    \end{equation}
    {\color{red} COMPLETE THIS EXAMPLE WHEN YOU CAN BE BOTHERED}
\end{example}
The following example provides a functional which also exhibits the Lavrentiev gap phenomenon, even though it is weakly coercive 
\begin{example}[Ball-Minze]
    {\color{red} ALSO DO THIS}
\end{example}

\section{Integral Side Constraints}
We begin this section by extending the Direct Method to the situation in which there is an additional constraint.
\begin{theorem}[Direct Method with Side Constraint]
    Let $X$ be a topological space. Let $\scrF \colon X \to \bbR \cup \set{\infty}$ be a functional which is coercive and lower semicontinuous, and let $\scrH \colon X \to \bbR \cup \set{\infty}$ be a continuous functional. Suppose there exists $u_0 \in X$ with $\scrH[u_0] = 0$. Then the variational problem 
    \begin{equation} \label{eq:constrainedMinimizationProblem}
        \left\{
        \begin{aligned}
            \text{minimize }   &\scrF[u] \text{ over } u \in X; \\
            \text{subject to } &\scrH[u] = 0
        \end{aligned}
        \right.
    \end{equation}
    has a solution.
\end{theorem}
\begin{proof}
    By assumption, $\scrH^{-1}[0]$ is nonempty. If $\scrF[u] = \infty$ for all $u \in \scrH^{-1}[0]$, the proof is trivial. Otherwise, choose a sequence $u_j \in \scrH^{-1}[0]$ with $\scrF[u_j] \rightarrow \alpha := \inf_{u \in \scrH^{-1}[0]} \scrF[u]$. Then $\scrF[u_j]$ is bounded from above, so by coercivity of $\scrF$, we can choose a subsequence of $u_j$ with $u_j \rightarrow u$ in $X$. By continuity of $\scrH$, we have $\scrH[u] = 0$. Moreover, by lower semicontinuity of $\scrF$, we have $\alpha \leq \scrF[u] \leq \liminf_{j \to \infty} \scrF[u_j] \leq \alpha$. Thus $\scrF[u] = \alpha$, and so $u$ is a minimizer of $\scrF$ over $\scrH^{-1}[0]$, i.e. a solution of (\ref{eq:constrainedMinimizationProblem}).
\end{proof} 

\begin{lemma}
    {\color{red} LEMMA IS QUESTIONABLE IN THE CASE $p > d$ TRY FIXING IT}
    Let $h \colon \Omega \times \bbR^m \to \bbR$ be a Carath\'eodory integrand. Fix $p \in [1,\infty]$, and suppose $h$ satisfies 
    \begin{enumerate}[label={\rm (\roman*)}]
        \item If $p < d$, then $h$ has $q$-growth for some $q < p^*$. That is, $\abs{ h(x,v) } \leq M(1 + \abs{v}^q)$ for some $M > 0$, a.e. $x \in \Omega$, and all $v \in \bbR^m$.
        \item If $p = d$, then $h$ has $q$-growth for some $q < \infty$.
    \end{enumerate}
    Then the corresponding integral functional 
    \begin{equation}
        \scrH[u] := \int_\Omega h(x,u(x)) \; \d x
    \end{equation}
    is weakly continuous on $W^{1,p}(\Omega;\bbR^m)$.
\end{lemma}
\begin{proof}
    Suppose $u_j \weak u$ in $W^{1,p}(\Omega;\bbR^m)$. Pick an arbitrary subsequence of $u_j$. For $p \leq d$, use Rellich-Kondrachov (theorem \ref{thm:rellichKondrachov}) to pass to a subsequence of $u_j$ with $u_j \rightarrow u$ in $L^q$. Pass to a further subsequence with $u_j \rightarrow u$ a.e. We have 
    \begin{equation}
        \abs{h(x,u_j(x))} \leq M(1 + \abs{u_j(x)}^q) \rightarrow M(1+\abs{u(x)}^q) \hspace{15pt} \text{pointwise a.e.}
    \end{equation}
    and
    \begin{equation} \begin{aligned}
        \int_\Omega M(1 + \abs{u_j(x)}^q) \; \d x &= M(\abs{\Omega} + \norm{u_j}_{W^{1,q}}^q) \\
                                                  &\rightarrow M(\abs{\Omega} + \norm{u}_{W^{1,q}}^q) \\
                                                  &= \int_\Omega M(1 + \abs{u}^q) \; \d x.
    \end{aligned} \end{equation}
    By Pratt's lemma \ref{lem:pratt}, 
    \begin{equation}
        \scrH[u_j] = \int_\Omega h(x,u_j(x)) \; \d x \rightarrow \int_\Omega h(x,u(x)) \; \d x = \scrH[u],
    \end{equation}
    which completes the proof for $p \leq d$, since this convergence holds for a subsequence of every subsequence of the original sequence $u_j$.

    For $p > d$, we can pick a subsequence with $u_j \rightarrow u$ uniformly. Since $h(x,\cdot)$ is continuous for a.e. $x \in \Omega$,
\end{proof}
We then get an existence theorem for constrained minimization problems:
\begin{theorem}
    Let $f \colon \Omega \times \bbR^m \times \bbR^{m \times d} \to \bbR$ and $h \colon \Omega \times \bbR^m \to \bbR$ be Carath\'eodory integrands such that 
    \begin{enumerate}[label={\rm (\roman*)}]
        \item $f$ is $p$-coercive for some $p \in (1,\infty)$.
        \item $f(x,v,\cdot)$ is convex for a.e. $x \in \Omega$ and all $v \in \bbR^m$.
        \item If $p < d$, $h$ has $q$-growth for some $q < p^*$, and if $p = d$, $h$ has $q$-growth for some $q < \infty$.
    \end{enumerate}
    Then there is a solution of the variational problem 
    \begin{equation}
        \left\{ 
        \begin{aligned}
            \text{minimize } \scrF[u] &= \int_\Omega f(x,u(x),\nabla u(x)) \; \d x \text{ over } u \in W^{1,p}(\Omega;\bbR^m), \\
            \text{subject to } u\vert_{\partial \Omega} &= g \\
            \text{and } \scrH[u] &= \int_\Omega h(x,u(x)) \; \d x = 0.
        \end{aligned}
        \right.
    \end{equation}
\end{theorem}

\chapter{Variations}

Suppose a functional $\scrF \colon W^{1,p}_g(\Omega;\bbR^m) \to \bbR$ has a minimizer $u_*$. Consider a path $t \mapsto u_t$ in $W_g^{1,p}$ with $u_0 = u_*$. If the map $t \mapsto \scrF[u_t]$ is differentiable at $t=0$, then its derivative must be zero by calculus. More generally, for $u \in W^{1,p}_g(\Omega;\bbR^m)$, define the \textit{first variation of} $\scrF$ \textit{at} $u$ to be the (partially defined) map $\delta\scrF[u] \colon C_c^\infty(\Omega;\bbR^m) \to \bbR$ given by 
\begin{equation}
    \delta\scrF[u][\psi] := \lim_{h \to 0} \frac{\scrF[u+h\psi] - \scrF[u]}{h} 
                          = \left. \frac{\mathrm{d}}{\mathrm{d}t} \scrF[u+t\psi] \right\vert_{t=0}.
\end{equation}
if it exists. That is, $\delta\scrF$ is the G\^ateaux derivative of $\scrF$. A function $u_* \in W^{1,p}_g(\Omega;\bbR^m)$ is called a \textit{critical point of} $\scrF$ if $\delta\scrF[u_*] = 0$.

\section{The Euler-Lagrange Equation}
Given a map $f \colon \Omega \times \bbR^m \times \bbR^{m \times d} \to \bbR$, define the \textit{directional derivatives of} $f$ \textit{at the point} $(x,v,A) \in \Omega \times \bbR^m \times \bbR^{m \times d}$ to be the partially defined maps $\D_2f(x,v,A) \colon \bbR^m \to \bbR$ and $\D_3f(x,v,A) \colon \bbR^{m \times d} \to \bbR$ given by 
\begin{align}
    \D_2f(x,v,A) \cdot w &:= \lim_{h \to 0} \frac{f(x,v+hw,A) - f(x,v,A)}{h}, \\
    \D_3f(x,v,A) : B     &:= \lim_{h \to 0} \frac{f(x,v,A+hB) - f(x,v,A)}{h},
\end{align}
whenever these limits exist. We also denote $\D_2$ by $\D_v$ and $\D_3$ by $\D_A$. The notations ``$\cdot$'' and ``$:$'' are merely suggestive, since $\D_v f(x,v,A)$ and $\D_A f(x,v,A)$ are not necessarily a vector or a matrix respectively. However, if $f$ is $C^1$ in the second argument, then $\D_v f(x,v,A)$ is given by the vector 
\begin{equation}
    \parens{ \frac{\partial f}{\partial v^i}(x,v,A) }^{i=1,\dots,m}.
\end{equation}
Similarly, if $f$ is $C^1$ in the third argument, then $\D_A f(x,v,A)$ is given by the matrix 
\begin{equation}
    \parens{ \frac{\partial f}{\partial \tensor{A}{_j^i}} }_{j=1,\dots,d}^{i=1,\dots,m}.
\end{equation}

\begin{theorem}
    Let $\Omega \subseteq \bbR^d$ be open with Lipschitz boundary, and let $f \colon \Omega \times \bbR^m \times \bbR^{m \times d} \to \bbR$ be a Carath\'eodory integrand which is $C^1$ in $v$ and $A$, and which satisfies the growth bound 
    \begin{equation} \label{eq:eulerLagrangeGrowthBound}
        \abs{\D_v f(x,v,A)}, \abs{\D_A f(x,v,A)} \leq C(1 + \abs{v}^p + \abs{A}^p),
    \end{equation}
    for some $p \in [1,\infty)$ and $C > 0$, a.e. $x \in \Omega$, and all $(v,A) \in \bbR^m \times \bbR^{m \times d}$. Let $\scrF$ be the corresponding integral functional defined by
    \begin{equation}
        \begin{aligned}
            \scrF[u] := \int_\Omega f(x,u(x),\nabla u(x)) \; \d x; \hspace{30pt} u \in W^{1,p}_g(\Omega;\bbR^m).
        \end{aligned} 
    \end{equation}
    Then 
    \begin{equation} \label{eq:firstVariationIntegralFormula}
        \delta\scrF[u][\psi] = \int_\Omega \D_A f(x,u_*(x),\nabla u_*(x)) : \nabla \psi + \D_v f(x,u_*(x),\nabla u_*(x)) \cdot \psi \; \d x.
    \end{equation}
    Consequently, $u_* \in W^{1,p}_g(\Omega;\bbR^m)$ is a critical point of $\scrF$ if and only if $u_*$ is a weak solution of the boundary value problem 
    \begin{equation} \label{eq:eulerLagrange}
        \left\{
        \begin{aligned}
            -\div[\D_A f(x,u(x),\nabla u(x))] + \D_v f(x,u,\nabla u(x)) &= 0 \text{ in } \Omega,         \\
                                                                      u &= g \text{ on } \partial\Omega.
        \end{aligned} 
        \right.
    \end{equation}
\end{theorem}
Equation (\ref{eq:eulerLagrange}) is called the \textit{Euler-Lagrange equation for} $\scrF$.
\begin{proof}
    Fix $\psi \in C_c^\infty(\Omega;\bbR^m)$. Then 
    \begin{equation} \begin{aligned}
        \frac{\scrF[u_* + h\psi] - \scrF[u_*]}{h} &= \int_\Omega \frac{f(x,u_* + \psi,\nabla u_* + h\nabla\psi
                                                                        - f(x,u_*,\nabla u_*)}{h} \; \d x               \\
                                                  &= \frac{1}{h} \int_\Omega \int_0^1 \frac{\d}{\d t} f(x,u_* + th\psi,
                                                                            \nabla u_* + th\nabla\psi \; \d t \, \d x   \\
                                                  &= \int_\Omega \int_0^1 \D_v f(x,u_* + th\psi,
                                                        \nabla u_* + th\nabla\psi ) \cdot \psi                          \\
                                                  &\hspace{30pt} + \D_A f(x,u_* + th\psi, \nabla u_* 
                                                        + th\nabla\psi ) : \nabla \psi \; \d t \, \d x.
    \end{aligned} \end{equation}
    The growth bound (\ref{eq:eulerLagrangeGrowthBound}) allows us to choose a dominating function for the the integral on the right. So taking the limit as $h \to 0$ in both the left and right hand side and applying the dominated convergence theorem, we find 
    \begin{equation} 
        \delta\scrF[u_*] = \int_\Omega \D_A f(x,u_*,\nabla u_*) \cdot \psi + \D_v f(x,u_*,\nabla u_*) : \nabla \psi \; \d x.
    \end{equation}
    This finishes the proof.
\end{proof}
\begin{remark}
    Suppose (\ref{eq:eulerLagrangeGrowthBound}) is replaced by the slightly stronger growth bound
    \begin{equation} \label{eq:eulerLagrangeStrongerGrowthBound}
        \abs{\D_v f(x,v,A)}, \abs{\D_A f(x,v,A)} \leq C(1 + \abs{v}^{p-1} + \abs{A}^{p-1}).
    \end{equation}
    Then $\delta\scrF[u][v]$ can be defined for $v \in W^{1,p}_0(\Omega;\bbR^m)$. Indeed, the integral formula (\ref{eq:firstVariationIntegralFormula}) for the first variation satisfies 
    \begin{equation} \begin{aligned}
        \abs{\delta\scrF[u][\psi]} &\leq \int_\Omega \abs{ \D_A f(x,u,\nabla u) } \abs{\psi} + \abs{ \D_v f(x,u,\nabla u) }                                           \abs{\nabla \psi} \; \d x                                                           \\
                                   &\leq \int_\Omega C(1 + \abs{u}^{p-1} + \abs{\nabla u}^{p-1})(\abs{\psi} 
                                       + \abs{\nabla\psi}) \; \d x                                                            \\
                                   &\leq C(1 + \norm{u}_{L^{p'(p-1)}}^{p-1} + \norm{\nabla u}_{L^{p'(p-1)}}^{p-1})
                                         (\norm{\psi}_{L^p} + \norm{\nabla\psi}_{L^p})                                        \\
                                   &\leq C(1 + \norm{u}_{L^p}^{p-1} + \norm{\nabla u}_{L^p}^{p-1})\norm{\psi}_{W^{1,p}}.
    \end{aligned} \end{equation}
    So $\delta\scrF[u]$ is bounded with respect to the $W^{1,p}$ norm, meaning we can use density of $C_c^\infty$ in $W^{1,p}_0$ to extend formula (\ref{eq:firstVariationIntegralFormula}).
\end{remark}
\begin{proposition} \label{prop:convexityImpliesSolutionsOfEulerLagrangeAreMinimizers}
    Let $\Omega \subseteq \bbR^d$ be open with Lipschitz boundary, and let $f \colon \Omega \times \bbR^m \times \bbR^{m \times d} \to \bbR$ be a Carath\'eodory integrand satisfying the stronger growth bound (\ref{eq:eulerLagrangeStrongerGrowthBound}), and such that the map $(v,A) \mapsto f(x,v,A)$ is convex for a.e. $x \in \Omega$. If $u_* \in W^{1,p}_g(\Omega;\bbR^m)$ is a critical point of the corresponding integral functional $\scrF$ defined by (\ref{eq:integralFunctionalWithFunctionDependence}), then $u_*$ is a minimizer of $\scrF$ over $W^{1,p}_g(\Omega;\bbR^m)$.
\end{proposition}
\begin{proof}
    Fix $v \in W^{1,p}_g(\Omega;\bbR^m)$. Define $h(t) := \scrF[u_* + t(v-u_*)]$. Then 
    \begin{equation}
        h'(0) = \delta\scrF[u_*][v-u_*] = 0.
    \end{equation}
    By convexity, $h(t) \geq h(0) + th'(0) = \scrF[u_*]$, and so $\scrF[v] = h(1) \geq \scrF[u_*]$.
\end{proof}

\begin{example}
    The Euler-Lagrange equation for the Dirichlet functional
    \begin{equation}
        \scrF[u] = \int_{\Omega} \frac{1}{2}\abs{\nabla u}^2 \; \d x
    \end{equation}
    is given by the \textit{Laplace equation} $-\Delta u = 0$. Note that the integrand satisfies the growth bound (\ref{eq:eulerLagrangeStrongerGrowthBound}) for $p=2$ and is strictly convex, so proposition \ref{prop:convexityImpliesSolutionsOfEulerLagrangeAreMinimizers} implies harmonic functions (i.e. solutions $u \in W^{1,2}_g(\Omega)$ of the Laplace equation) are minimizers of $\scrF$, and proposition \ref{prop:strictConvexityImpliesUniqueness} implies this solution is the unique minimizer over $W^{1,p}_g$. In particular, solutions of the Laplace equation with given boundary values are unique.

    The same is true for the Dirichlet functional with an additional term:
    \begin{equation}
        \scrF[u] = \int_\Omega \frac{1}{2}\abs{\nabla u}^2 - b \cdot u \; \d x.
    \end{equation}
    The Euler-Lagrange equation in this instance is the \textit{Poisson equation} $- \Delta u = b$.
\end{example}

An important special case are integrands of the form $f(x,v,A) = f(x,A) = \frac{1}{2}A:S(x)A$, where $S(x)$ is a symmetric $(2,2)$-tensor. That is, $S(x) \colon \bbR^{m \times d} \to \bbR^{m \times d}$ is a linear map, and the components of $SA$ are given by 
\begin{equation}
    \tensor{(S(x)A)}{_i^j} = \tensor{S}{_i^j_l^k}(x)\tensor{A}{_k^l}
\end{equation}
where the Einstein summation convention is in force. In particular,
\begin{equation}
    A:S(x)A = \sum_{i=1}^d\sum_{j=1}^m \tensor{A}{_i^j}\tensor{S}{_i^j_l^k}(x)\tensor{A}{_k^l}.
\end{equation}
In this situation, the Euler-Lagrange equation of the corresponding functional $\scrF$ is 
\begin{equation}
    -\div[S\nabla u] = 0.
\end{equation}
If $S(x)$ is positive definite for all $x$ (this occurs if, e.g., $f(x,\cdot)$ is convex), then the Euler-Lagrange equation above is an elliptic PDE.

Beyond this point, we consider the situation $p=2$. A \textit{strong solution} of (\ref{eq:eulerLagrange}) is a function $u \in W^{2,2}(\Omega;\bbR^m)$ (or, more generally, in $(W^{1,2} \cap W^{2,2}_{\rm loc})(\Omega;\bbR^m))$ which satisfies the Euler-Lagrange equation pointwise a.e. If we also have that $u \in C^2(\Omega;\bbR^m) \cap C^0(\overline{\Omega};\bbR^m)$ is a strong solution, then it is also a \textit{classical solution} (i.e. all the derivatives of $u$ in the Euler-Lagrange equation can be interpreted to be usual derivatives). It would be nice to know that a sufficiently differentiable weak solution of the Euler-Lagrange equation is actually a strong solution. This is indeed the case:
\begin{proposition}
    Suppose $f \in C^2(\Omega \times \bbR^m \times \bbR^{m \times d};\bbR)$, and $u \in (W^{1,2} \cap W^{2,2}_{\rm loc})(\Omega;\bbR^m)$ is a weak solution of (\ref{eq:eulerLagrange}). Then $u$ is a strong solution. 
\end{proposition}
\begin{proof}
    Starting from (\ref{eq:firstVariationIntegralFormula}) with $\delta\scrF[u] = 0$, we use the divergence theorem to find
    \begin{equation}
        \int_\Omega ( -\div[\D_A f(x,u_*(x),\nabla u_*(x))] + \D_v f(x,u_*(x),\nabla u_*(x)) )\cdot \psi \; \d x = 0
    \end{equation}
    for all $\psi \in C_c^\infty(\Omega;\bbR^m)$. The proof is finished by the following lemma.
\end{proof}
\begin{lemma}[Fundamental Lemma of the Calculus of Variations]
    Let $\Omega \subseteq \bbR^d$ be open. Suppose $g \in L^1_{\rm loc}(\Omega)$ satisfies 
    \begin{equation}
        \int_\Omega g\psi \; \d x = 0 \hspace{20pt} \text{for all } \psi \in C_c^\infty(\Omega).
    \end{equation}
    Then $g = 0$ a.e.
\end{lemma}
\begin{proof}
    Let $\epsilon > 0$. Extend $g$ by zero to $\bbR^d$. Fix a compact set $K \subseteq \bbR^d$, and choose $h \in C_c^\infty(\bbR^d)$ such that $\norm{ g - h }_{L^1(K)} < \frac{\epsilon}{2}$. Let $(\eta_\delta)_{\delta > 0}$ be a family of mollifiers, and choose $\delta > 0$ such that $\phi := \eta_\delta \ast \sgn{h}$ satisfies $\norm{ \phi - \sgn{h} }_{L^1(\bbR^d) } < \frac{\epsilon}{2(1+\norm{h}_{L^\infty(\bbR^d)})}$. Then 
    \begin{equation} \begin{aligned}
        \norm{g}_{L^1(K)} &\leq \norm{ g - h }_{L^1(K)} + \norm{h}_{L^1(K)}    \\
                          &=    \norm{g-h}_{L^1(K)} + \int_K h \sgn{h} \; \d x \\
                          &=    \norm{g-h}_{L^1(K)} + \int_K g\phi \; \d x + \int_K (h-g)\phi \; \d x + \int_K h(\sgn{h} - \phi) \;        \d x                                           \\
                          &\leq \norm{g-h}_{L^1(K)} + 0  + \norm{g-h}_{L^1(K)} + \norm{h}_{L^\infty}\norm{\sgn{h}-\phi}_{L^1} \\
                          &<    \epsilon,
    \end{aligned} \end{equation}
    where we use the fact that $\norm{\phi}_{L^\infty} \leq \norm{\eta_\delta}_{L^1} = 1$. Since $\epsilon > 0$ was arbitrary, we conclude $\norm{g}_{L^1(K)} = 0$, and so $g = 0$ a.e. on $K$. But $K$ was also arbitrary, so $g = 0$ a.e. on $\Omega$.
\end{proof}

\section{Regularity of Minimizers}
We have seen that a critical point $u$ of an integral functional $\scrF$ which has $W^{2,2}_{\rm loc}$ regularity is a strong solution of the Euler-Lagrange equation for $\scrF$. A question to ask is for which integral functionals $\scrF$ does being a critical point guarantee such regularity? We will answer this question for a certain class of integral functionals. Let 
\begin{equation}
    \scrF[u] = \int_\Omega f(\nabla u) \; \d x \quad u \in W^{1,2}(\Omega;\bbR^m).
\end{equation}
We say $\scrF$ is a \textit{regular variational integral} if $f$ is $C^\infty$, and there exist $\mu,M > 0$ such that
\begin{equation}
    \mu \abs{B}^2 \leq \D^2 f(A)[B,B] \leq M \abs{B}^2 \quad \text{for all } A,B \in \bbR^{m \times d}.
\end{equation}
Immediately from the growth condition, we have 
\begin{align}
    \abs{\D^2 f(A)[B_1,B_2]} &\leq M \abs{B_1}\abs{B_2}, \\
    \abs{\D f(A_1) - \D f(A_2)} &\leq M\abs{A_1 - A_2}, \\
    \abs{\D f(A)} &\leq M(1 + \abs{A}),
\end{align}
for some possibly different constants $M$.

Our goal of this section is to prove the following regularity theorem:
\begin{theorem}
    Let $\scrF$ be a regular variational integral with integrand $f$, and suppose $u_* \in W^{1,2}(\Omega;\bbR^m)$ is a cricical point of $\scrF$. Then $u_*$ is twice weakly differentiable, and moreover, for any open ball $B(x_0,3r) \subseteq \Omega$, the following \textit{Caccioppoli inequality} holds:
    \begin{equation}
        \int_{B(x_0,r)} \abs{\nabla^2 u_*}^2 \; \d x \leq \parens{\frac{2M}{\mu}}^2 \int_{B(x_0,3r)} \frac{\abs{\nabla u_* - [\nabla u_*]_{B(x_0,3r)}}^2}{r} \; \d x,
    \end{equation}
    where $[\nabla u_*]_{B(x_0,3r)} = \fint_{B(x_0,3r)} u \; \d x$ denotes the average of $\nabla u_*$ over $B(x_0,3r)$.

    Consequently, $u_*$ has $W^{2,2}_{\rm loc}$ regularity, and therefore satisfies the Euler-Lagrange equation strongly.
\end{theorem}
The proof of this theorem will rely on using difference quotients to emulate the second derivative of $u_*$, and proving estimates to show these emulations converge to a true second derivative. Some revision on difference quotients is needed first. 

Let $u \colon \Omega \to \bbR^m$ be a function. For $k \in \set{1,\dots,d}$, the $k$\textit{th difference quotient} with \textit{height} $h \in \bbR \setminus \set{0}$ is defined by 
\begin{equation}
    \D^h_k u(x) := \frac{u(x+he_k) - u(x)}{h}
\end{equation}
whenever this is well-defined. The matrix $\D^hu(x) \in \bbR^{m \times d}$ is given by $\tensor{(\D^hu(x))}{_i^j} = \D^h_i u^j(x)$.
\begin{lemma}
    Let $D \Subset \Omega \subseteq \bbR^d$ be open, let $p \in [1,\infty)$, and $u \in L^p(\Omega;\bbR^m)$.
    \begin{enumerate}[label={\rm (\arabic*)}]
        \item If $u \in W^{1,p}(\Omega;\bbR^m)$, then $\norm{\D^h_k u}_{L^p(D;\bbR^m)} \leq \norm{\partial_k u}_{L^p(\Omega;\bbR^m)}$ for all $k \in \set{1,\dots,d}$ and $0 < \abs{h} \leq d(D,\partial \Omega)$.
        
        \item Suppose $p > 1$, and there exists $0 < \delta < d(D,\partial \Omega)$ and $C > 0$ such that there is the bound $\norm{\D^h_k u}_{L^p(D;\bbR^m)} \leq C$ for all $0 < \abs{h} < \delta$. Then $\partial_k u$ exists on $D$, and $\norm{\partial_k u}_{L^p(\Omega;\bbR^m)} \leq C$. In particular, if this holds for all $k \in \set{1,\dots,d}$ and $D \Subset \Omega$, then $u \in W^{1,p}_{\rm loc}(\Omega;\bbR^m)$.
    \end{enumerate}
\end{lemma}
\begin{proof}
    \begin{enumerate}[label = (\arabic*)]
        \item For $x \in D$ and $0 < \abs{h} < d(D,\partial \Omega)$, we have
        \begin{equation} \begin{aligned}
            \D^h_k u(x) &= \frac{u(x+he_k) - u(x)}{h} \\
                        &= \frac{1}{h} \int_0^1 \frac{\d}{\d t} u(x+the_k) \; \d t \\
                        &= \int_0^1 \partial_k u(x+the_k) \; \d x.
        \end{aligned} \end{equation}
        By Jensen's inequality for the probability space $([0,1],\scrL^1)$ and convexity of $\abs{\cdot}^p$, we have 
        \begin{equation} \begin{aligned}
            \fhnorm{\D^h_k u}_{L^p(D;\bbR^m)}^p &= \int_D \abs{ \int_0^1 \partial_k u(x+the_k) \; \d t}^p \; \d x \\
                                              &\leq \int_D \int_0^1 \abs{\partial_k u(x+the_k)}^p \; \d t \, \d x \\
                                              &\leq \norm{\partial_k u}_{L^p(\Omega;\bbR^m)}^p,
        \end{aligned} \end{equation}
        as required.

        \item The set $\set{\D^h_k u : 0 < \abs{h} < \delta}$ is bounded in the reflexive space $L^p(D;\bbR^m)$, so the Banach-Alaoglu theorem provides the existence of a sequence $h_j \downarrow 0$ and a function $v_k \in L^p(D;\bbR^m)$ such that $\D^{h_j}_k u \weak v_k$. By lower semicontinuity of norms with respect to weak convergence, we have $\norm{v_k}_{L^p(D;\bbR^m)} \leq C$. Now, fix $\phi \in C_c^\infty(D;\bbR^m)$. Then 
        \begin{equation} \begin{aligned}
            \int_D \partial_k \phi \cdot u \; \d x &= \lim_{j \to \infty} \int_D \D^{-h_j}_k \phi \cdot u \; \d x \\
                                                   &= - \lim_{j \to \infty} \int_D \phi \cdot \D^{h_j}_k u \; \d x \\
                                                   &= - \int_D \phi \cdot v_k \; \d x.
        \end{aligned} \end{equation}
        So $v_k$ is the weak derivative $\partial_k u$.
    \end{enumerate}
\end{proof}

{\color{red} complete this section}

\section{Lagrange Multipliers}
{\color{red} complete this section}

\section{Invariances and Noether's Theorem}
{\color{red} complete this section}

\chapter{Young Measures}

Suppose $V_j \weak V$ in $L^2$. Given a Carathe\'eodory integrand $f$, we would like to know the limit of $\int_\Omega f(x,V_j(x)) \; \d x$. In general, this cannot be $\int_\Omega f(x,V(x)) \; \d x$. For example, [{\color{red} complete this}]

\section{The Fundamental Theorem}
Fix a bounded Lipschitz domain $\Omega \subseteq \bbR^d$. A family $\nu = (\nu)_{x \in \Omega}$ of probability measures on $\bbR^N$ is called \textit{weakly* measurable} if, for all $f \in C_0(\Omega \times \bbR^N)$, the function 
\begin{equation}
    x \mapsto \angles{f(x,\cdot), \nu_x} := \int_{\bbR^N} f(x,A) \; \d \nu_x(A)
\end{equation}
is Lebesgue measurable. To be able to go any further, we need to be able to extend measurability to more general functions.
\begin{proposition}
    Let $\nu = (\nu_x)_{x \in \Omega}$ be a weakly* measurable family of probability measures on $\bbR^N$. Then, for any Carath\'eodory integrand $f \colon \Omega \times \bbR^N \to \bbR$, the function 
    \begin{equation}
        x \mapsto \angles{f(x,\cdot),\nu_x}
    \end{equation}
    is Lebesgue measurable.
\end{proposition}
\begin{proof}
    {\color{red} maybe prove if you want}
\end{proof}
Because of how useful it is, we will be using this proposition in the rest of the chapter without warning.

For $p \in [1,\infty)$, we say $\nu$ is an $L^p$-\textit{Young measure} if it is weakly* measurable, and 
\begin{equation}
    \llangle \abs{\cdot}^p, \nu \rrangle := \int_\Omega \int_{\bbR^N} \abs{A}^p \; \d\nu_x(A) \, \d x < \infty.
\end{equation}
Similarly, we say $\nu$ is an $L^\infty$-\textit{Young measure} if it is weakly* measurable, and there exists a compact set $K \subseteq \bbR^N$ such that $\supp{\nu_x} \subseteq K$ for a.e. $x \in \Omega$. For $p \in [1,\infty]$, the set of all such $L^p$-Young measures is denoted $\bfY^p(\Omega;\bbR^N)$.

A subset $A \subseteq \bfY^p(\Omega;\bbR^N)$ is said to be \textit{bounded} if 
\begin{equation}
    \sup_{\nu \in A} \aangles{\abs{\cdot}^p,\nu} < \infty \quad \text{for } p < \infty,
\end{equation}
and for $p = \infty$, there exists a compact set $K \subseteq \bbR^N$ such that for all $\nu \in A$, $\supp{\nu_x} \subseteq K$ for a.e. $x \in \Omega$.

The most important theorem in Young measure theory is the following:
\begin{theorem}[Fundamental Theorem of Young Measure Theory] \label{thm:fundamentalTheoremOfYoungMeasures}
    For $p \in [1,\infty]$, let $V_j \in L^p(\Omega;\bbR^N)$ be a bounded sequence. Then there exists a subsequence of $V_j$, and a Young measure $\nu \in \mathbf{Y}^p(\Omega;\bbR^N)$ such that
    \begin{equation} \label{eq:youngLimitForContinuousFunctions}
        \lim_{j \to \infty} \int_\Omega f(x,V_j(x)) \; \d x = \int_\Omega \int_{\bbR^N} f(x,A) \; \d\nu_x(A) \, \d x
    \end{equation}
    for all $f \in C_0(\Omega \times \bbR^N)$.
\end{theorem}
We call $\nu$ the Young measure \textit{generated by} $V_j$, and we write $V_j \young \nu$. Our goal will be to prove the fundamental theorem by showing a more general compactness result for Young measures.

Given a sequence $\nu_j$ of weakly* measurable families of probability measures, we say $\nu_j$ \textit{converges weakly*} to another weakly* measurable family $\nu$ if $\fhaangles{f,\nu_j} \rightarrow \fhaangles{f,\nu}$ for all $f \in C_0(\Omega \times \bbR^N)$. As usual, we write $\nu_j \weakstar \nu$. 
The proof of theorem \ref{thm:fundamentalTheoremOfYoungMeasures} is an immediate consequence of the following Young measure compactness and approximation lemmas.
\begin{lemma}[Compactness] \label{lem:youngMeasureCompactness}
    Let $p \in [1,\infty]$, and let $\nu^{(j)} \in \bfY^p(\Omega;\bbR^N)$ be a bounded sequence of Young measures. Then there is a subsequence of $\nu^{(j)}$ and a weakly* measurable family $\nu = (\nu_x)_{x \in \Omega}$ such that $\nu^{(j)} \weakstar \nu$. 
\end{lemma}
\begin{lemma}[Approximation] \label{lem:youngMeasureApproximation}
    Let $\nu^{(j)} \in \bfY^p(\Omega;\bbR^N)$ be a bounded sequence of Young measures with $\nu^{(j)} \weakstar \nu$. Let $f \colon \Omega \times \bbR^N \to \bbR$ be a Carath\'eodory integrand satisfying
    \begin{enumerate}[label={\rm (\roman*)}]
        \item (Uniform integrability)
        \begin{equation}
            \sup_{j \in \bbN} \int_\Omega \abs{\fhangles{f(x,\cdot),\nu^{(j)}_x}} \; \d x < \infty,
        \end{equation}
        
        \item (Equiintegrability)
        \begin{equation}
            \lim_{h \to \infty} \sup_{j \in \bbN} \fhaangles{ \abs{f} \bbone_{ \{(x,A) \in \Omega \times \bbR^N : \abs{f(x,A) \geq h}\} }, \nu^{(j)} } = 0.
        \end{equation}
    \end{enumerate}
    Then 
    \begin{equation}
        \lim_{j \to \infty} \fhaangles{f,\nu^{(j)}} = \fhaangles{f,\nu}.
    \end{equation}
\end{lemma}
\begin{corollary} \label{cor:lowerSemicontinuityOfYoungMeasures}
    For $p \in [1,\infty]$, let $\nu_j \in \bfY^p(\Omega;\bbR^N)$ be a bounded sequence converging weakly* to $\nu$. Let $f \colon \Omega \times \bbR^N \to \bbR$ be a Carath\'eodory integrand. Then the lower semicontinuity estimate 
    \begin{equation}
        \aangles{f,\nu} \leq \liminf_{j \to \infty} \aangles{f,\nu^{(j)}}
    \end{equation}
    holds. In particular, $\nu$ is an $L^p$-Young measure.
\end{corollary}
\begin{proof}
    For $h > 0$, define $f_h(x,A) := \min\{f(x,A),h\}$. Then (i) and (ii) of the approximation lemma hold, so
    \begin{equation}
        \liminf_{j \to \infty} \aangles{f,\nu^{(j)}} \geq \lim_{j \to \infty} \aangles{f_h,\nu^{(j)}}
                                                               = \aangles{f_h,\nu}.
    \end{equation}
    By the monotone convergence theorem, $\aangles{f_h,\nu} \rightarrow \aangles{f,\nu}$ as $h \to \infty$. The final statement follows by considering $f(x,A) = \abs{A}^p$ for $p < \infty$, and for $p = \infty$ doing some cool shit {\color{red} do this cool shit}
\end{proof}

The proof of the fundamental theorem is now immediate:
\begin{proof}[Proof of theorem \ref{thm:fundamentalTheoremOfYoungMeasures}]
    Define the \textit{elementary Young measure} $\delta[V_j] \in \bfY^P(\Omega;\bbR^N)$ by $\delta[V_j]_x := \delta_{V_j(x)}$. Then 
    \begin{equation}
        \fhaangles{ f,\delta[V_j] } = \int_\Omega f(x,V_j(x)) \; \d x
    \end{equation}
    for all $f \in C_0(\Omega \times \bbR^N)$. It is easy to see that $\nu^{(j)} := \delta[V_j]$ if a bounded sequence in $\bfY^p(\Omega;\bbR^N)$. Lemma \ref{lem:youngMeasureCompactness} and corollary \ref{cor:lowerSemicontinuityOfYoungMeasures} then complete the proof.
\end{proof}

\begin{proof}[Proof of lemma \ref{lem:youngMeasureApproximation}]
    {\color{red} prove using scorza-dragoni if u want :)}
\end{proof}
\begin{proof}[Proof of lemma \ref{lem:youngMeasureCompactness}] \renewcommand{\qedsymbol}{}
    Define measures $\mu^{(j)} \in \scrM(\Omega \times \bbR^N)$ by 
    \begin{equation}
        \angles{f,\mu} = \int_\Omega \int_{\bbR^N} f(x,A) \; \d\nu^{(j)}_x(A) \, \d x
    \end{equation}
    for $f \in C_0(\Omega \times \bbR^N)$. This defines a measure since $\scrM(\Omega \times \bbR^N) \cong C_0(\Omega \times \bbR^N)^*$. Note that 
    \begin{equation}
        \abs{\fhangles{f,\mu^{(j)}}} \leq \abs{\Omega} \norm{f}_{L^\infty}
    \end{equation}
    for all $f \in C_0(\Omega \times \bbR^N)$. The sequence $\mu^{(j)}$ is therefore bounded in $C_0(\Omega \times \bbR^N)^*$, which, by the Banach-Alaoglu theorem, implies the existence of a subsequence and a measure $\mu \in \scrM(\Omega \times \bbR^N)$ such that $\mu^{(j)} \weakstar \mu$. To continue, we will need the following disintegration theorem.
\end{proof}
\begin{theorem}[Disintegration of Measures]
    Let $\Omega \subset \bbR^d$ be open and $\mu \in \scrM(\Omega \times \bbR^N)$ a Radon measure. Then there exists a weakly* measurable family $\nu = (\nu)_{x \in \Omega}$ of probability measures such that 
    \begin{equation}
        \angles{f,\mu} = \int_\Omega \int_{\bbR^N} f(x,A) \; \d\nu_x(A) \, \d\kappa(x)
    \end{equation}
    for all $f \in C_0(\Omega \times \bbR^N)$, where $\kappa \in \scrM(\Omega)$ is given by $\kappa(B) = \mu(B \times \bbR^N)$ for $B \subseteq \Omega$ a Borel set.
\end{theorem}
\begin{proof}[Proof of lemma continued]
    Using the disintegration theorem, let $\nu$ be the weakly* measurable family of probability measures corresponding to $\mu$. We need to show $\kappa = \scrL^d \restrict \Omega$. Suppose $p < \infty$, and let $U \subseteq \Omega$ be open. Then by lower semicontinuity of weak* convergence of measures, we have 
    \begin{equation}
        \kappa(U) = \mu(U \times \bbR^N) \leq \liminf_{j \to \infty} \mu^{(j)}(U \times \bbR^N) = \abs{U}.
    \end{equation}
    On the other hand, for $K \subseteq \Omega$ compact, upper semicontinuity gives us 
    \begin{equation} \begin{aligned}
        \mu(K \times \overline(B(0,R))) &\geq \limsup_{j \to \infty} \mu^{(j)}(K \times \overline{B(0,R)}) \\
                                        &=    \int_K \int_{\overline{B(0,R)}} \; \d\nu^{(j)}_x \, \d x \\
                                        &\geq \int_K \int_{\overline{B(0,R)}} 1 - \frac{\abs{A}^p}{R^p} \; \d\nu^{(j)}_x(A) \, \d x \\
                                        &\geq \abs{K} - \frac{1}{R^p} \sup_{j \in \bbR^N} \aangles{ \abs{\cdot}^p,\nu^{(j)} } \\
    \end{aligned} \end{equation}
    Taking $R \to \infty$ on both sides and using continuity of $\mu$, we conclude $\kappa(K) \geq \abs{K}$. Since $\kappa$ is a Radon measure, it follows that $\kappa = \scrL^d \restrict \Omega$.
\end{proof}

Let's consider some examples. Note that to prove $V_j \young \nu$, it suffices to show 
\begin{equation}
    \lim_{j \to \infty} \int_\Omega \phi(x) h(V_j(x)) \; \d x = \int_\Omega \phi(x) \angles{h,\nu_x} \; \d x
\end{equation}
for all $\phi \in C_0(\Omega)$ and $h \in C_0(\bbR^N)$. Indeed, functions of the form $(x,A) \mapsto \phi(x)h(A)$ are dense in $C_0(\Omega \times \bbR^N)$.

A Young measure $\nu = (\nu_x)_{x \in \Omega}$ is \textit{homogeneous} if $x \mapsto \nu_x$ is constant a.e.

\begin{example}
    \begin{enumerate}[label=(\arabic*)]
        \item On $\Omega = (0,1)$, define $u := \bbone_{(0,\frac{1}{2})} - \bbone_{(\frac{1}{2},1)}$, and extend this to $\bbR$ periodically. Define $u_j(x) := u(jx)$ for $x \in \Omega$. We claim $u_j$ generates the homogeneous $L^\infty$-Young measure $\nu = \frac{1}{2}(\delta_{+1} + \delta_{-1}) \in \bfY^\infty(\Omega)$. Inded, fix $\phi \in C_0(\Omega)$ and $h \in C_0(\bbR^N)$. Then, since $h$ is bounded,
        \begin{equation} \begin{aligned}
            \int_\Omega \phi(x) h(u_j(x)) \; \d x &= \sum_{k=0}^{j-1} \int_{k/j}^{(k+1)/j} \phi(x) h(u_j(x)) \; \d x \\
                                                &= \sum_{k=0}^{j-1} \int_{k/j}^{(k+1)/j} \phi\parens{\frac{k}{j}} h(u_j(x)) \; \d x + \frac{1}{j} O\parens{\omega\parens{\frac{1}{j}}} \\
                                                &= \sum_{k=0}^{j-1} \frac{1}{j} \phi\parens{\frac{k}{j}} \int_0^1 h(u(y)) \; \d y + \frac{1}{j} O\parens{\omega\parens{\frac{1}{j}}} \\
                                                &= \sum_{k=0}^{j-1} \frac{1}{j} \phi\parens{\frac{k}{j}} \int_\bbR h \; \d\nu + \frac{1}{j} O\parens{\omega\parens{\frac{1}{j}}} \\
                                                &\to \int_\Omega \phi(x) \; \d x \int_\bbR h(A) \; \d\nu(A) \\
                                                &= \int_\Omega \int_{\bbR} \phi(x) h(A) \; \d\nu_x(A) \, \d x,
        \end{aligned} \end{equation}
        as required. In the third line, we made the change of variables 
        \begin{equation}
            x = \frac{k}{j} + \frac{1}{j}y,
        \end{equation}
        and we also used the fact that $\phi$ has modulus of continuity $\omega$.

        \item Let $\Omega \subseteq \bbR^2$ be a bounded Lipschitz domain, and let $A,B \in \bbR^{2 \times 2}$ be \textit{rank-one connected}, in the sense that $B-A$ has rank at most 1. We may then write $B - A = an^\T$ for some $a,n \in \bbR^2$. Let $\theta_A, \theta_B \in (0,1)$ be such that $\theta_A + \theta_B = 1$. For $x \in \bbR^2$, define 
        \begin{equation}
            u(x) := Ax + \parens{ \int_0^{x \cdot n} \bbone_{{k \in \bbZ} [k,k+\theta_B))}(t) \; \d t } a.
        \end{equation}
        Then, for $j \in \bbN$, define $u_j \in W^{1,\infty}(\Omega;\bbR^2)$ by $u_j(x) := j^{-1} u(jx)$. Then 
        \begin{equation}
            \nabla u_j(x) = A + \bbone_{{k \in \bbZ} [k,k+\theta_B))}(jx \cdot n) an^\T.
        \end{equation}
        We now claim $\nabla u_j$ generates the homogeneous Young measure $\nu = \theta_A \delta_A + \theta_B \delta_B \in \bfY^\infty(\Omega;\bbR^{2 \times 2})$.
    \end{enumerate}
\end{example}

Suppose a bounded sequence $V_j \in L^p(\Omega;\bbR^N)$ generates the Young measure $\nu \in \bfY^p(\Omega;\bbR^N)$. Given a Carath\'eodory integrand $f \colon \Omega \times \bbR^N \to \bbR$, the approximation lemma tells us that if 
\begin{enumerate}[label=(\roman*)]
    \item (Uniform integrability)
    \begin{equation}
        \sup_{j \in \bbN} \int_\Omega \abs{f(x,V_j(x))} \; \d x < \infty,
    \end{equation}
    and 
    \item (Equiintegrability)
    \begin{equation}
        \lim_{h \to \infty} \sup_{j \in \bbN} \int_{\set{\abs{f(\cdot,V_j)} \geq h}} \abs{f(x,V_j(x))} \bbone_{\abs{f} \geq h}(x,A) \; \d x = 0,
    \end{equation}
\end{enumerate}
then
\begin{equation}
    \lim_{j \to \infty} \int_\Omega f(x,V_j(x)) \; \d x = \int_\Omega \int_{\bbR^N} f(x,A) \; \d\nu_x(A) \, \d x.
\end{equation}
This property along with corollary \ref{cor:lowerSemicontinuityOfYoungMeasures} will be important in the rest of these notes. Also note that this convergence works fine for vector-valued integrands $f$, by considering each component separately.

\section{Young Measures and Convergence}
Given a Young measure $\nu \in \bfY^p(\Omega;\bbR^N)$, its \textit{barycenter} $[\nu] \in L^p(\Omega;\bbR^N)$ is defined by 
\begin{equation}
    [\nu](x) = [\nu_x] := \int_\Omega A \; \d\nu_x(A).
\end{equation}

\begin{lemma} \label{lem:barycenterConvergence}
    For $p \in (1,\infty]$, let $V_j \in L^p(\Omega;\bbR^N)$ be a bounded sequence generating a Young measure $\nu \in \bfY^p(\Omega;\bbR^N)$. Then $V_j \weak [\nu]$ if $p < \infty$, and $V_j \weakstar [\nu]$ if $p = \infty$.
\end{lemma}
\begin{proof}
    First, we take $p = \infty$. Take a test function $\phi \in L^1(\Omega)$. Since $V_j \in L^\infty(\Omega;\bbR^N)$ is bounded, the function $\phi V_j$ is clearly uniformly integrable and equiintegrable. We can then take $f(x,A) := A$ to see 
    \begin{equation}
        \lim_{j \to \infty} \int_\Omega \phi(x) V_j(x) \; \d x = \int_\Omega \phi(x) [\nu](x) \; \d x.
    \end{equation}
    It follows that $V_j \weakstar [\nu]$ in $L^\infty$.

    Now take $p < \infty$. By Banach-Alaoglu, $V_j$ is weakly precompact in $L^p$, so by Dunford-Pettis, {\color{red} finish}
\end{proof}
Note this lemma fails for $p = 1$, a counterexample is the concentrating sequence $V_j = j\bbone_{(0,\frac{1}{j})}$.

The following lemma demonstrates a useful technique we can use to prove convergence in measure for a number of functions, so long as we have a Young measure at hand.
\begin{lemma} \label{lem:ymConvergeInMeas}
    Let $\nu \in \bfY^p(\Omega;\bbR^N)$, $p \in [1,\infty]$ be generated by a bounded sequence $V_j \in L^p(\Omega;\bbR^N)$. Then 
    \begin{enumerate}[label={\rm (\roman*)}]
        \item For any compact set $K \subseteq \bbR^N$, we have $d(V_j,K) \to 0$ in measure if and only if $\supp{\nu_x} \subseteq K$ for a.e. $x \in \Omega$.
        \item For any $V \in L^p(\Omega;\bbR^N)$, we have $V_j \to V$ in measure if and only if $\nu_x = \delta_{V(x)}$ for a.e. $x \in \Omega$.
    \end{enumerate}
\end{lemma}
\begin{proof}
    Let $f \colon \Omega \times \bbR^N \to [0,1]$ be a Carath\'eodory integrand. Fix $\epsilon \in (0,1)$. Since $f$ is bounded, $f(\cdot,V_j)$ is uniformly integrable and equiintegrable, so by the approximation lemma and the Markov inequality, 
    \begin{equation} \begin{aligned}
        \limsup_{j \to \infty} \abs{\set{ x \in \Omega : f(x,V_j(x)) \geq \epsilon }} &\leq \lim_{j \to \infty} \frac{1}{\epsilon} \int_\Omega f(x,V_j(x)) \; \d x \\
                                                                                      &= \frac{1}{\epsilon} \int_\Omega \int_{\bbR^N} f(x,A) \; \d\nu_x(A) \, \d x,
    \end{aligned} \end{equation}
    so if $\angles{f(x,\cdot),\nu_x} = 0$ for a.e. $x \in \Omega$, then $f(\cdot,V_j) \to 0$ in measure. Conversely, we also have 
    \begin{equation} \begin{aligned}
        \int_\Omega \int_{\bbR^N} f(x,A) \; \d\nu_x(A) \, \d x &= \lim_{j \to \infty} \int_\Omega f(x,V_j(x)) \; \d x \\
                                                               &\leq \epsilon\abs{\Omega} + \abs{\set{x \in \Omega : f(x,V_j(x)) \geq \epsilon}}.
    \end{aligned} \end{equation}
    So if $f(\cdot,V_j) \to 0$ in measure, then $\angles{f(x,\cdot),\nu_x} = 0$ for a.e. $x \in \Omega$.

    To finish the proof, it remains to select appropriate integrands $f$. For (i), take 
    \begin{equation}
        f(x,A) := \frac{d(A,K)}{1 + d(A,K)}.
    \end{equation}
    Then $d(V_j,K) \to 0$ in measure if and only if $f(\cdot,V_j) \to 0$ in measure, which, by the above work, is equivalent to 
    \begin{equation}
        0 = \angles{f(x,\cdot),\nu_x} = \int_{\bbR^N} \frac{d(A,K)}{1 + d(A,K)} \; \d\nu_x(A) \quad \text{for a.e. } x \in \Omega.
    \end{equation}
    This is equivalent to $\supp{\nu_x} \subseteq K$ for a.e. $x \in \Omega$.

    For (ii), take 
    \begin{equation}
        f(x,A) := \frac{\abs{A - V(x)}}{1 + \abs{A - V(x)}}.
    \end{equation}
    Then $V_j \to V$ in measure if and only if $f(\cdot,V_j) \to 0$ in measure, which is therefore equivalent to 
    \begin{equation}
        0 = \angles{f(x,\cdot),\nu_x} = \int_{\bbR^N} \frac{\abs{A - V(x)}}{1 + \abs{A - V(x)}} \; \d\nu_x(A) \quad \text{for a.e. } x \in \Omega.
    \end{equation}
    This is equivalent to $\nu_x = \delta_{V(x)}$ for a.e. $x \in \Omega$.
\end{proof}

\section{Gradient Young Measures}
Let $\nu \in \bfY^p(\Omega;\bbR^{m \times d})$ be a Young measure. If there exists a bounded sequence $u_j \in W^{1,p}(\Omega;\bbR^m)$ such that $\nabla u_j$ generates $\nu$, then we say $\nu$ is a \textit{gradient Young measure}, and write $\nu \in \bfGY^p(\Omega;\bbR^{m \times d})$.
\begin{lemma} \label{lem:gradientYM}
    For $p \in (1,\infty]$, let $\nu \in \bfGY^p(\Omega;\bbR^{m \times d})$, and let $u \in W^{1,p}(\Omega;\bbR^m)$ be an \textit{underlying deformation} of $u$, which is to say $[\nu] = \nabla u$. Then there exists a bounded sequence $u_j \in W^{1,p}(\Omega;\bbR^m)$ such that $\supp(u_j - u) \Subset \Omega$ and $\nabla u_j$ generates $\nu$. Furthermore, if $p < \infty$, then $\abs{\nabla u_j}^p$ is equiintegrable.
\end{lemma}

Homogeneous gradient Young measures satisfy a certain averaging property, characterized by the following lemmas.
\begin{lemma}
    For $p \in [1,\infty]$, let $\nu \in \bfGY^p(\Omega;\bbR^{m \times d})$ have underlying deformation $u \in W^{1,p}(\Omega;\bbR^m)$ such that $u$ has linear boundary values. That is, $u \vert_{\partial\Omega}(x) = Ax$ for all $x \in \partial\Omega$ and some linear map $A \colon \bbR^d \to \bbR^m$. Then, for any bounded Lipschitz domain $D \subseteq \bbR^d$, there exists a homogeneous gradient Young measure $\overline{\nu} \in \bfGY^p(D;\bbR^{m \times d})$ such that 
    \begin{equation}
        \int_{\bbR^{m \times d}} h(A) \; \d\overline{\nu}(A) = \dashint_\Omega \int_{\bbR^{m \times d}} h(A) \; \d\nu_x(A) \, \d x.
    \end{equation}
    for any continuous $h \colon \bbR^{m \times d} \to \bbR$ with $p$-growth.
\end{lemma}
An important special case:
\begin{lemma}[Riemann-Lebesgue]
    For $p \in [1,\infty]$, let $u \in W^{1,p}(\Omega;\bbR^m)$ have linear boundary values. Then there is a homogeneous gradient Young measure $\overline{\delta[\nabla u]} \in \bfGY^p(\Omega;\bbR^{m \times d})$ such that 
    \begin{equation}
        \int_{\bbR^{m \times d}} h(A) \; \d\overline{\delta[\nabla u]} = \dashint_\Omega h(\nabla u(x)) \; \d x
    \end{equation}
    for all continuous $h \colon \bbR^{m \times d} \to \bbR$ with $p$-growth.
\end{lemma}

\chapter{Quasiconvexity}
\section{Quasiconvex Functions}
Let $h \colon \bbR^{m \times d} \to \bbR$ be locally bounded and Borel measurable. We say $h$ is \textit{quasiconvex} if 
\begin{equation}
    h(A) \leq \dashint_{B(0,1)} h(A + \nabla \psi(x)) \; \d x
\end{equation}
for all $A \in \bbR^{m \times d}$ and $\psi \in W_0^{1,\infty}(B(0,1);\bbR^m)$, where $B(0,1) \subseteq \bbR^d$ is the unit open ball. We can give a physical interpretation of quasiconvexity: suppose 
\begin{equation}
    \scrF[y] := \int_{B(0,1)} h(\nabla y(x)) \; \d x
\end{equation}
models the elastic energy of deforming the ball $B(0,1)$ via a deformation $y \in W^{1,\infty}(B(0,1);\bbR^m)$. If $y$ is affine, i.e. $y(x) = x_0 + Ax$ for a linear map $A$, we would expect this deformation to cost no energy. Thus if we add an internal deformation $\psi \in W_0^{1,\infty}(B(0,1);\bbR^m)$, we should expect $y + \psi$ should cost more energy than just $y$. That is,
\begin{equation}
    \abs{B(0,1)} h(A) = \int_{B(0,1)} h(A) \; \d x = \scrF[y] \leq \scrF[y + \psi] = \int_{B(0,1)} h(A + \nabla\psi(x)) \; \d x.
\end{equation}
In particular, if $h$ is quasiconvex, then this physical condition is necessarily true.

Let's show that quasiconvexity is actually a notion of convexity:
\begin{lemma}
    Suppose $h \colon \bbR^{m \times d} \to \bbR$ is locally bounded and convex. Then $h$ is quasiconvex.
\end{lemma}
\begin{proof}
    Choose $V \in L^1(B(0,1);\bbR^m)$ with $\int_{B(0,1)} V(x) \; \d x = 0$, and fix $A \in \bbR^{m \times d}$. Define a probability measure $\mu \in \scrM^1(\bbR^{m \times d})$ by 
    \begin{equation}
        \angles{f,\mu} := \dashint_{B(0,1)} f(A + V(x)) \; \d x \quad \text{for } f \in C_0(\bbR^{m \times d}).
    \end{equation}
    By Jensen's inequality,
    \begin{equation} \begin{aligned}
        h(A) &= h\parens{ \int_{\bbR^{m \times d}} \id \; \d\mu } \\
             &\leq \int_{\bbR^{m \times d}} h \; \d\mu \\
             &= \dashint_{B(0,1)} h(A + V(x)) \; \d x.
    \end{aligned} \end{equation}
    Taking $V := \nabla \psi$ for $\psi \in W^{1,\infty}_0(B(0,1);\bbR^m)$, the proof is finished.
\end{proof}

Some more aspects surrounding the definition of quasiconvexity are encapsulated in the following lemma.
\begin{lemma} \label{lem:propertiesOfQuasiconvexity}
    \begin{enumerate}[label={\rm (\arabic*)}]
        \item Let $\Omega \subseteq \bbR^d$ be a bounded Lipschitz domain. Suppose $h \colon \bbR^{m \times d} \to \bbR$ is locally bounded and satisfies 
        \begin{equation} \label{eq:quasiconvexity}
            h(A) \leq \dashint_\Omega h(A + \nabla \psi(x)) \; \d x
        \end{equation}
        for all $A \in \bbR^{m \times d}$ and $\psi \in W_0^{1,\infty}(\Omega;\bbR^m)$. Then $h$ is quasiconvex.

        \item Suppose $h$ is locally bounded and has $p$-growth for some $p \in [1,\infty)$. If $h$ satisfies (\ref{eq:quasiconvexity}) for all $\psi \in W_0^{1,p}(\Omega;\bbR^m)$, then $h$ is quasiconvex.
    \end{enumerate}
\end{lemma}  
\begin{proof}
    (1) Given $\psi \in W_0^{1,p}(\Omega;\bbR^m)$ and another bounded Lipschitz domain $\widetilde{\Omega}$, we will show there exists $\widetilde{\psi} \in W_0^{1,p}(\widetilde{\Omega};\bbR^m)$ such that for all $A \in \bbR^{m \times d}$ and measurable $h \colon \bbR^{m \times d} \to \bbR$, we have 
    \begin{equation}
        \dashint_{\Omega} h(A + \nabla\psi(x)) \; \d x = \dashint_{\widetilde{\Omega}} h(A + \nabla\widetilde{\psi}(x)) \; \d x,
    \end{equation}
    in the sense that if one of the above integrals exists and is finite, then both exist and are finite, and the above equality holds. Property (i) will then follow from taking $\widetilde{\Omega} = B(0,1)$ and $p = \infty$

    We use the Vitali covering theorem to write 
    \begin{equation}
        \widetilde{\Omega} = Z \cup \bigcup_{i=1}^\infty \Omega(a_i,r_i)
    \end{equation}
    for some $a_i \in \widetilde{\Omega}$, $r_i > 0$ and $Z \subseteq \widetilde{\Omega}$ with $\abs{Z} = 0$, where $\Omega(a,r) = a + r\Omega$. Given $x \in \Omega(a_i,r_i)$, we define 
    \begin{equation}
        \widetilde{\psi}(x) = r_i \psi\parens{\frac{x-a_i}{r_i}}.
    \end{equation}
    This defines an element of $L^p(\widetilde{\Omega})$. Evidently, $\widetilde{\psi}$ is weakly differentiable on each $\Omega(a_i,r_i)$ with weak gradient $\nabla\psi$. Thus $\widetilde{\psi}$ is weakly differentiable on $\widetilde{\Omega}$, and since $\nabla\widetilde{\psi} = \nabla\psi$, we see $\widetilde{\psi}$ lies in $W^{1,p}(\widetilde{\Omega};\bbR^m)$. 
    {\color{red} finish}  
\end{proof}

A weaker notion of convexity is the following: a locally bounded Borel-measurable function $h \colon \bbR^{m \times d} \to \bbR$ is \textit{rank-one convex} if 
\begin{equation}
    h(\theta A + (1-\theta)B) \leq \theta h(A) + (1-\theta) h(B).
\end{equation}
for all $A,B \in \bbR^{m \times d}$ with $\rank(A-B) \leq 1$ and $\theta \in [0,1]$.
That is, $h$ is convex along any rank-1 line.
\begin{theorem}
    If $h$ is quasiconvex, then it is rank-one convex.
\end{theorem}
\begin{proof}
    Fix $A,B \in \bbR^{m \times d}$ with $B - A = an^\T$ for some $a \in \bbR^m \setminus \set{0}$ and $n \in S^{d-1}$. Choose $\theta \in (0,1)$. Define $Q \subseteq \bbR^d$ to be an open cube with two faces orthogonal to $n$, and with unit volume. Write $F := \theta A + (1-\theta)B$, and define the \textit{laminate} $u_j \in W_0^{1,\infty}(Q;\bbR^m)$ as follows: set
    \begin{equation}
        \phi_0(t) :=
        \begin{cases}
            -(1-\theta)t & t \in [0,\theta], \\
            \theta(t-1)  & t \in (\theta,1].
        \end{cases}
    \end{equation}
    Then define 
    \begin{equation}
        u_j(x) := Fx + \frac{1}{j} \phi_0(jx \cdot n - \lfloor jx \cdot n \rfloor)a.
    \end{equation}
    (cf. proposition \ref{prop:wlscImpliesConvex}). Now, $u_j$ has gradient 
    \begin{equation}
        \nabla u_j(x) = 
        \begin{cases}
            F - (1-\theta) an^\T = A & jx - \lfloor jx \cdot n \rfloor \in [0,\theta], \\
            F + \theta an^\T = B     & jx - \lfloor jx \cdot n \rfloor \in (\theta,1].
        \end{cases}
    \end{equation}
    So 
    \begin{equation} \label{eq:averageOfLaminate}
        \lim_{j \to \infty} \dashint_{Q} h(\nabla u_j(x)) \; \d x = \theta h(A) + (1-\theta) h(B).
    \end{equation}
    We also have $u_j \weakstar F$ in $W^{1,\infty}$, so by the Rellich-Kondrachov theorem, $u_j \to F$ uniformly.

    Now, take a sequence $\rho_k \in C_c^\infty(Q;[0,1])$ with $\abs{Q \setminus \set{\rho_k = 1}} \to 0$ as $k \to \infty$. Define $v_{j,k} \in W^{1,\infty}(Q;\bbR^m)$ by 
    \begin{equation}
        v_{j,k}(x) := \rho_k(x)u_j(x) + (1-\rho_k(x)) Fx.
    \end{equation}
    Then $v_{j,k} = F$ on $\partial Q$, and
    \begin{equation}
        \nabla v_{j,k}(x) = \rho_k(x) \nabla u_j(x) + (1-\rho_k(x)) F + (u_j(x) - Fx) \nabla\rho_k(x).
    \end{equation}
    For each $k \in \bbN$, we may then estimate 
    \begin{equation}
        \limsup_{j \to \infty} \norm{\nabla v_{j,k}}_{L^\infty} \leq \limsup_{j \to \infty} \norm{\nabla u_j}_{L^\infty} + \abs{F} < \infty,
    \end{equation}
    noting that $\norm{u_j - F}_{L^\infty} \to 0$ as $j \to \infty$, and $\norm{\nabla u_j}_{L^\infty} \leq \max\set{\abs{A},\abs{B}}$ for all $j \in \bbN$. Choose, therefore, a sequence $j(k) \in \bbN$ with $\norm{\nabla v_{j(k),k}}_{L^\infty} < \infty$ independently of $k$. Since $h$ is locally bounded, there exists $C > 0$ such that 
    \begin{equation}
        \norm{h(\nabla u_{j(k)})}_{L^\infty} + \norm{h(\nabla v_{j(k),k})}_{L^\infty} \leq C \quad \text{for all } k \in \bbN.
    \end{equation}
    We then have the estimate 
    \begin{equation} \begin{aligned}
        \lim_{k \to \infty} \int_Q \abs{h(\nabla v_{j(k),k}) - h(\nabla u_{j(k)})} \; \d x
        &\leq \lim_{k \to \infty} \int_{Q \setminus \set{\rho_k = 1}} \abs{h(\nabla v_{j(k),k})} + \abs{h(\nabla u_{j(k)})} \; \d x \\
        &\leq C\abs{Q \setminus \set{\rho_k = 1}} \\
        &= 0.
    \end{aligned} \end{equation}
    By the above calculation and (\ref{eq:averageOfLaminate}), it follows by quasiconvexity (noting $v_{j,k} - F \in W_0^{1,\infty}(Q;\bbR^m)$) that 
    \begin{equation} \begin{aligned}
        h(F) &\leq \lim_{k \to \infty} \dashint_Q h(F + (\nabla v_{j(k),k}(x) - F)) \; \d x \\
             &= \lim_{k \to \infty} \dashint_Q h(\nabla u_{j(k)}(x)) \; \d x \\
             &= \theta h(A) + (1-\theta) h(B),
    \end{aligned} \end{equation}
    as required.
\end{proof}
For $d=1$ or $m=1$, convexity, quasiconvexity, and rank-one convexity are all equivalent. For $d,m \geq 2$, however, quasiconvexity is strictly weaker than convexity. For example, we will see in the next section that the determinant and all smaller minors (besides $(1 \times 1)$-minors) are quasiconvex, but not convex. The following example is standard in the literature:
\begin{example}[Alibert-Dacorogna-Marcellini]
    For $d=m=2$ and $\gamma \in \bbR$, define $h_\gamma \colon \bbR^{2 \times 2} \to \bbR$ by 
    \begin{equation}
        h_\gamma(A) := \abs{A}^2 \parens{ \abs{A}^2 - 2\gamma \det{A} }.
    \end{equation}
    Then 
    \begin{itemize}
        \item $h_\gamma$ is convex if and only if $\abs{\gamma} \leq \frac{2\sqrt{2}}{3}$,
        \item $h_\gamma$ is rank-one convex if and only if $\abs{\gamma} \leq \frac{2}{\sqrt{3}}$,
        \item $h_\gamma$ is quasiconex if and only if $\abs{\gamma} \leq \gamma_{\rm QC}$ for some currently unknown $\gamma_\mathrm{QC} \in \left(1,\frac{2}{\sqrt{3}}\right]$.
    \end{itemize}
\end{example}

We end this section with a couple of results about rank-one convexity:
\begin{lemma}
    Let $h \colon \bbR^{m \times d} \to \bbR$ be arnk-one convex, and suppose there exists $M > 0$ and $p \in [1,\infty)$ such that 
    \begin{equation}
        h(A) \leq M(1 + \abs{A}^p) \quad \text{for all } A \in \bbR^{m \times d}.
    \end{equation}
    Then $h$ has $p$-growth.
\end{lemma}
\begin{lemma}
    Let $h \colon \bbR^{m \times d} \to \bbR$ be rank-one convex. Then it is locally Lipschitz continuous. Furthermore, if $h$ has $p$-growth with growth constant $M > 0$, then there exists $C = C_{d,m} > 0$ such that 
    \begin{equation}
        \abs{h(A) - h(B)} \leq CM(1 + \abs{A}^{p-1} + \abs{B}^{p-1})\abs{A-B} \quad \text{for all } A,B \in \bbR^{m \times d}.
    \end{equation}
    In particular, a rank-one convex function with linear growth is uniformly Lipschitz continuous.
\end{lemma}

\section{Null-Lagrangians}
The \textit{null-Lagrangians} are the class of locally bounded functions $h \colon \bbR^{m \times d} \to \bbR$ such that the integral $\int_\Omega h(\nabla u(x)) \; \d x$ only depends on the boundary values of $u$.

An \textit{ordered } $r$-\textit{multiindex} is an $r$-tuple $\alpha = (\alpha_1,\dots,\alpha_r)$ with $\alpha_1 < \cdots < \alpha_r$. Fix the rank $r \in \set{1,\dots,\min\set{d,m}}$, and let $\alpha$ and $\beta$ be ordered $r$-multiindices, with $\alpha_i \in \set{1,\dots,m}$, and $\beta_i \in \set{1,\dots,d}$ for all $i$. Write $A_\beta^\alpha$ for the $(r \times r)$-matrix with the $\alpha$ rows and $\beta$ columns. We say the function $M = M_\beta^\alpha \colon \bbR^{m \times d} \to \bbR$ given by 
\begin{equation}
    M(A) := \det{A_\beta^\alpha}
\end{equation}
is an $(r \times r)$-\textit{minor}. We say $r$ is its \textit{rank}. The following lemma says all minors are null-Lagrangians:
\begin{lemma} \label{lem:minorsAreNullLagrangians}
    Let $M \colon \bbR^{m \times d} \to \bbR$ be an $(r \times r)$-minor. For $p \in [r,\infty)$, suppose $u,v \in W^{1,p}(\Omega;\bbR^m)$ are such that $u = v$ on $\partial\Omega$ (in the sense that $v - u \in W_0^{1,p}(\Omega;\bbR^m)$). Then 
    \begin{equation}
        \int_\Omega M(\nabla u(x)) \; \d x = \int_\Omega M(\nabla v(x)) \; \d x.
    \end{equation}
\end{lemma}
\begin{proof}
    Suppose first that $u,v$ are in $C^\infty(\overline{\Omega})$ with $\supp(u-v) \Subset \Omega$. Without loss of generality, we may take $M$ to be the minor of the first $r$ rows and columns (such an $M$ is called a \textit{principal minor}). Using notation from differential geometry (including the Einstein summation convention), we then have 
    \begin{equation} \begin{aligned}
        \d u^1 \wedge \cdots \d u^r \wedge \d x^{r+1} \wedge \cdots \wedge \d x^d 
        &= \pdv{u^1}{x^{j_1}} \cdots \pdv{u^r}{x^{j_r}} \, \d x^{j_1} \wedge \cdots \d x^{j_r} \wedge \d x^{r+1} \wedge \cdots \wedge \d x^d \\
        &= \sum_{\sigma \in S_r} (-1)^{\abs{\sigma}} \pdv{u^1}{x^{\sigma_1}} \cdots \pdv{u^r}{x^{\sigma_r}} \, \d x^1 \wedge \cdots \wedge \d x^d \\
        &= M(\nabla u) \, \d x^1 \wedge \d x^d,
    \end{aligned} \end{equation}
    where $S_r$ is the group of permutations of $\set{1,\dots,r}$. Now note that 
    \begin{equation}
        \d u^1 \wedge \cdots \d u^r \wedge \d x^{r+1} \wedge \cdots \wedge \d x^d 
        = \d(u^1 \d u^2 \wedge \cdots \d u^r \wedge \d x^{r+1} \wedge \cdots \wedge \d x^d ),
    \end{equation}
    so by Stokes' theorem, we have 
    \begin{equation} \begin{aligned}
        \int_\Omega M(\nabla u) \; \d x &= \int_\Omega \d u^1 \wedge \cdots \d u^r \wedge \d x^{r+1} \wedge \cdots \wedge \d x^d \\
        &= \int_{\partial \Omega} u^1 \d u^2 \wedge \cdots \d u^r \wedge \d x^{r+1} \wedge \cdots \wedge \d x^d \\
        &= \int_{\partial \Omega} v^1 \d v^2 \wedge \cdots \d v^r \wedge \d x^{r+1} \wedge \cdots \wedge \d x^d \\
        &= \int_\Omega M(\nabla v) \; \d x,
    \end{aligned} \end{equation}
    finishing the proof in the smooth case.

    In the non-smooth case, take $\epsilon > 0$. By Hadamard's inequality, $M(A) \leq \abs{A}^r$ for any $A \in \bbR^{m \times d}$, so Pratt's lemma (lemma \ref{lem:pratt}) immediately implies the function $u \mapsto \int_\Omega M(\nabla u) \; \d x$ is strongly continuous as a map $W^{1,p}(\Omega) \to \bbR$. Let $\delta > 0$ be such that $\norm{\widetilde{u} - \widetilde{v}}_{W^{1,p}} < \delta$ implies $\abs{\int_\Omega M(\nabla u) - M(\nabla v) \; \d x} < \frac{\epsilon}{2}$.
    We may find $\widetilde{u} \in W^{1,p}(\Omega) \cap C^\infty(\overline{\Omega})$ such that $\norm{\widetilde{u} - u}_{W^{1,p}} < \frac{\delta}{2}$. Furthermore, since $v = u$ on $\partial\Omega$, there exists $\psi \in C_c^\infty(\Omega)$ such that $\norm{(v - u) - \psi}_{W^{1,p}} < \frac{\delta}{2}$. Define $\widetilde{v} := \widetilde{u} + \psi \in C^\infty(\overline{\Omega})$. Then 
    \begin{equation}
        \norm{\widetilde{v} - v}_{W^{1,p}} \leq \norm{(v - u) - \psi}_{W^{1,p}} + \norm{\widetilde{u} - u}_{W^{1,p}} < \delta.
    \end{equation}
    Note furthermore that $\widetilde{v} - \widetilde{u} = \psi$, so $\supp(\widetilde{v} - \widetilde{u}) \Subset \Omega$. It follows that 
    \begin{equation} \begin{aligned}
        \abs{\int_\Omega M(\nabla u) \; \d x - \int_\Omega M(\nabla v) \; \d x} 
        &\leq \abs{\int_\Omega M(\nabla u) \; \d x - \int_\Omega M(\nabla \widetilde{u}) \; \d x} \\
        &\quad + \abs{\int_\Omega M(\nabla \widetilde{u}) \; \d x - \int_\Omega M(\nabla \widetilde{v}) \; \d x} \\
        &\quad + \abs{\int_\Omega M(\nabla \widetilde{v}) \; \d x - \int_\Omega M(\nabla v) \; \d x} \\
        &< \frac{\epsilon}{2} + 0 + \frac{\epsilon}{2} \\\
        &= \epsilon.
    \end{aligned} \end{equation}
    Since $\epsilon > 0$ was arbitrary, this finishes the proof.
\end{proof}

\begin{corollary}
    All minors $M \colon \bbR^{m \times d} \to \bbR$ are \textit{quasiaffine}, which means $M$ and $-M$ are quasiconvex.
\end{corollary}
\begin{proof}
    By Hadamard's inequality, $M$ has $r$-growth, where $r$ is the rank of $M$. Take a test function $\psi \in W_0^{1,r}(B(0,1);\bbR^m)$. Then, for any $A \in \bbR^{m \times d}$, the map $A + \psi$ is in $W^{1,r}(B(0,1);\bbR^m)$, and $\psi = A$ on $\partial B(0,1)$. By lemma \ref{lem:minorsAreNullLagrangians}, we have 
    \begin{equation}
        M(A) = \dashint_{B(0,1)} M(A) \; \d x = \dashint_{B(0,1)} M(A + \nabla \psi) \; \d x,
    \end{equation}
    so part (2) of lemma \ref{lem:propertiesOfQuasiconvexity} implies $M$ is quasiconvex. Putting a minus sign in front of everything tells us that $-M$ is also quasiconvex.
\end{proof}

Minors actually enjoy a certain weak continuity property:
\begin{lemma} \label{lem:minorWeakCty}
    Let $M \colon \bbR^{m \times d} \to \bbR$ be a rank $r$ minor, and for $p \in (r,\infty)$, let $u_j \in W^{1,p}(\Omega;\bbR^m)$ be a sequence with $u_j \weak u$ in $W^{1,p}$. Then $M(\nabla u_j) \weak M(\nabla u)$ in $L^{p/r}$. Similarly, if $u_j \in W^{1,\infty}(\Omega;\bbR^m)$ is a sequence with $u_j \weakstar u$ in $W^{1,\infty}$, then $M(\nabla u_j) \weakstar M(\nabla u)$ in $L^\infty$.
\end{lemma}
\begin{proof}
    {\color{red} can u do it using differential formz}
\end{proof}

\section{Quasiconvexity and Young Measures}
In this section, we will start to prove analogous results about convex functions for quasiconvex functions, using the machinery of Young measures. First, we have a Jensen-type inequality:
\begin{lemma} \label{lem:quasiconvexJensen}
    For $p \in (1,\infty)$, let $\nu \in \bfGY^p(B(0,1);\bbR^{m \times d})$ be a homogeneous gradient Young measure, and let $h \colon \bbR^{m \times d} \to \bbR$ be a quasiconvex function with $p$-growth. Then 
    \begin{equation}
        h([\nu]) \leq \int_{\bbR^{m \times d}} h \; \d\nu.
    \end{equation}
    The same result hold for $p=\infty$ without any growth assumption on $h$.
\end{lemma}
\begin{proof}
    By lemma \ref{lem:gradientYM}, we can find a sequence $u_j \in W^{1,p}_{[\nu]}(B(0,1);\bbR^m)$ such that $\abs{\nabla u_j}^p$ is equiintegrable and $\nabla u_j \young \nu$. By quasiconvexity, we have 
    \begin{equation}
        h([\nu]) \leq \dashint_{B(0,1)} h(\nabla u_j) \; \d x.
    \end{equation}
    Now, the sequence $h(\nabla u_j)$ is uniformly integrable and equiintegrable by assumption, so we may take the limit on the right hand side to obtain our result.

   For $p=\infty$, it is easy to check uniformly integrability and equiintegrability.
\end{proof}
\begin{corollary}
    For $p \in (1,\infty]$, let $\nu \in \bfGY^p(\Omega;\bbR^{m \times d})$ be a homogeneous gradient Young measure, and let $h \colon \bbR^{m \times d} \to \bbR$ be quasiaffine with $p$-growth. Then 
    \begin{equation}
        h([\nu]) = \int_{\bbR^{m \times d}} h \; \d\nu.
    \end{equation}
\end{corollary}

We now turn to considering the minimization problem (\ref{eq:integralMinimizationProblemWithoutFunctionDependence}) and, in particular, showing lower semicontinuity in the quasiconvex case. Assume $f$ has $p$-growth for some $p \in (1,\infty)$. Suppose $u_j \weak u$ in $W^{1,p}(\Omega;\bbR^m)$. Up to subsequence, we may assume $\nabla u_j$ generates $\nu \in \bfGY^p(\Omega;\bbR^{m \times d})$. If we assume the sequence $f(\cdot,\nabla u_j)$ is equiintegrable, then we may take the Young measure limit to find 
\begin{equation}
    \lim_{j \to \infty} \scrF[u_j] = \int_\Omega \int_{\bbR^{m \times d}} f(x,A) \; \d\nu_x(A) \, \d x.
\end{equation}
In order to show semicontinuity, it then suffices to show the inequality 
\begin{equation}
    f(x,\nabla u(x)) \leq \int_{\bbR^{m \times d}} f(x,A) \; \d\nu_x(A)
\end{equation}
for a.e. $x \in \Omega$. We have seen this in the homogeneous case when $f$ is quasiconvex, so our aim is to show a.e. $\nu_x$ is a homogeneous gradient Young measure in its own right. We do this using a blowup technique.
\begin{lemma} \label{lem:homogenizingGradientYM}
    For $p \in [1,\infty)$, let $\nu \in \bfGY^p(\Omega;\bbR^{m \times d})$ be a gradient Young measure. Then, for a.e. $x_0 \in \Omega$, the measure $\nu_{x_0}$ is a homogeneous gradient Young measure in $\bfGY^p(B(0,1);\bbR^{m \times d})$.
\end{lemma}
\begin{proof}
    Fix a bounded sequence $u_j \in W^{1,p}(\Omega;\bbR^m)$ such that $\nabla u_j$ generates $\nu$. Choose countable dense families $h_k \in C_0(\bbR^{m \times d})$ and $\phi_k \in C_0(\Omega)$. Then the family $\phi_k \otimes h_k$ is dense in $C_0(\Omega \times \bbR^{m \times d})$. Let $x_0 \in \Omega$ be a Lebesgue point of all the $L^1$ functions $x \mapsto \angles{h_k,\nu_x}$. That is,
    \begin{equation}
        \lim_{r \downarrow 0} \int_{B(0,1)} \abs{\angles{h_k,\nu_{x_0 + ry}} - \angles{h_k,\nu_{x_0}}} \; \d y = 0.
    \end{equation}
    This holds for a.e. $x_0 \in \Omega$. For $y \in B(0,1)$, define 
    \begin{equation}
        v_j^{(r)}(y) := \frac{u_j(x_0 + ry) - [u_j]_{B(x_0,r)}}{r}.
    \end{equation}
    Then 
    \begin{equation} \begin{aligned}
        \int_{B(0,1)} \phi_k(y) h_k(\nabla v_j^{(r)}(y)) \; \d y &= \int_{B(0,1)} \phi_k(y) h_k(\nabla u_j(x_0 + ry)) \; \d y \\
        &= \frac{1}{r^d} \int_{B(x_0,r)} \phi_k\parens{ \frac{z - x_0}{r} } h_k(\nabla u_j(z)) \; \d z.
    \end{aligned} \end{equation}
    Taking limits, we find 
    \begin{equation} \begin{aligned}
        \lim_{r \downarrow 0} \lim_{j \to \infty} \frac{1}{r^d} \int_{B(x_0,r)} \phi_k\parens{ \frac{z - x_0}{r} } h_k(\nabla u_j(z)) \; \d z
        &= \lim_{r \downarrow 0} \frac{1}{r^d} \int_{B(x_0,r)} \phi_k\parens{ \frac{z-x_0}{r} } \angles{ h_k, \nu_z } \; \d z \\
        &= \lim_{r \downarrow 0} \int_{B(0,1)} \phi_k(y) \angles{ h_k, \nu_{x_0 + ry} } \; \d y \\
        &= \int_{B(0,1)} \phi_k(y) \angles{ h_k, \nu_{x_0} } \; \d y.
    \end{aligned} \end{equation}
    Furthermore, we have 
    \begin{equation}
        \int_{B(0,1)} \abs{\nabla v_j^{(r)}(y)}^p \; \d y = \int_{B(0,1)} \abs{ \nabla u_j(x_0 + ry) }^p \; \d y
        = \frac{1}{r^d} \int_{B(x_0,r)} \abs{ \nabla u_j(z) }^p \; \d z.
    \end{equation}
    For fixed $r$, the latter integral is uniformly bounded in $j$. Now, up to subsequence, the measures $\abs{ \nabla u_j }^p \scrL^d \restrict \Omega$ converge weakly* to some finite measure $\lambda$. Assume $x_0$ also satisfies 
    \begin{equation}
        \lim_{r \downarrow 0} \frac{\lambda(\overline{B(x_0,r)})}{r^d} < \infty,
    \end{equation}
    which holds for a.e. $x_0 \in \Omega$ by the Besicovitch differentiation theorem. We then have 
    \begin{equation}
        \limsup_{r \downarrow 0} \lim_{j \to \infty} \int_{B(0,1)} \abs{\nabla v_j^{(r)}(y)}^p \; \d y < \infty.
    \end{equation}
    Since $[v_j^{(r)}]_{B(0,1)} = 0$, the Poincar\'e inequality yields a diagonal sequence $w_n := v_{j(n)}^{(r(n))}$ which is bounded in $W^{1,p}(B(0,1);\bbR^m)$ and such that 
    \begin{equation}
        \lim_{n \to \infty} \int_{B(0,1)} \phi_k(y) h_k(\nabla w_n(y)) \; \d y = \int_{B(0,1)} \phi_k(y) \angles{h_k,\nu_{x_0}} \; \d y.
    \end{equation}
    It follows that $\nabla w_n \young \nu_{x_0}$.
\end{proof}
Semicontinuity then follows:
\begin{theorem}[Morrey, Acerbi-Fusco]
    For $p \in (1,\infty)$, let $f \colon \Omega \times \bbR^{m \times d} \to [0,\infty)$ be a Carath\'eodory integrand with $p$-growth and such that $f(x,\cdot)$ is quasiconvex for a.e. $x \in \Omega$. Then the associated integral functional $\scrF$ is weakly lower semicontinuous on $W^{1,p}(\Omega;\bbR^m)$.
\end{theorem}
\begin{proof}
    Let $u_j$ be a sequence in $W^{1,p}(\Omega;\bbR^m)$ converging weakly to $u$. Choose an arbitrary subsequence, and pass to a further subsequence with $\nabla u_j \young \nu \in \bfGY^p(\Omega;\bbR^{m \times d})$. Then $[\nu] = \nabla u$. Now, corollary \ref{cor:lowerSemicontinuityOfYoungMeasures}, lemma \ref{lem:quasiconvexJensen}, and lemma \ref{lem:homogenizingGradientYM} combine to imply 
    \begin{equation} \begin{aligned}
        \liminf_{j \to \infty} \int_\Omega f(x,\nabla u_j(x)) \; \d x 
        &\geq \int_\Omega \int_{\bbR^{m \times d}} f(x,A) \; \d\nu_x(A) \, \d x \\
        &\geq \int_\Omega f(x,\nabla u(x)) \; \d x,
    \end{aligned} \end{equation}
    which is precisely the estimate to be proved.
\end{proof}
The Direct Method then ensures that if $\scrF$ is an integral functional whose integrand has $p$-growth, $p$-coercivity, and quasiconvexity, then it has a minimizer over $W^{1,p}(\Omega;\bbR^m)$ with $u \vert_{\partial \Omega} = g$.

Proposition \ref{prop:wlscImpliesConvex} has a big generalization for $d,m \neq 1$:
\begin{proposition}
    Let $f \colon \bbR^{m \times d} \to \bbR$ be continuous and with $p$-growth. If the associated functional
    \begin{equation}
        \scrF[u] := \int_\Omega f(\nabla u(x)) \; \d x, \quad u \in W^{1,p}(\Omega;\bbR^m)
    \end{equation}
    is weakly lower semicontinuous, then $f$ is quasiconvex.
\end{proposition}
\begin{proof}
    {\color{red} do u wanna do this}
\end{proof}


\section{Integrands with $u$-dependence}


\section{Regularity of Minimizers}


\section{Rigidity for Gradients}

Is every Young measure also a gradient Young measure? For inhomogeneous Young measures, the answer is false, since lemma \ref{lem:barycenterConvergence} implies the barycenter of a gradient Young measure is also a gradient. So a Young measure of the form $\delta[V]$ for some $V$ which is not a gradient cannot be a gradient Young measure.

Now, we turn to homogeneous Young measures. Fix $A,B \in \bbR^{m \times d}$ and $\theta \in (0,1)$. We know by a previous example that the homogeneous Young measure 
\begin{equation}
    \nu = \theta \delta_A + (1-\theta) \delta_B
\end{equation}
is a gradient Young measure for $\rank(A - B) \leq 1$. We investigate the case $\rank(A - B) \geq 2$ using the following rigidity theorem:
\begin{theorem}[Ball-James]
    Let $\Omega \subseteq \bbR^d$ be open, bounded, and connected, and let $A, B \in \bbR^{m \times d}$.
    \begin{enumerate}[label = {\rm (\roman*)}]
        \item Suppose $u \in W^{1,\infty}(\Omega;\bbR^m)$ satisfies the \textit{exact two-gradient inclusion} $\nabla u \in \set{A,B}$ a.e. in $\Omega$.
        \begin{enumerate}[label = {\rm (\alph*)}]
            \item If $\rank(A-B) \geq 2$, then $\nabla u = A$ a.e., or $\nabla u = B$ a.e.
            \item Suppose $\Omega$ is additionally convex. If $B - A = an^\T$ for some $a \in \bbR^m$ and $n \in S^{d-1}$, then there exists a Lipschitz function $h \colon \bbR \to \bbR$ with $h' \in \set{0,1}$ a.e., and $v_0 \in \bbR^m$ such that 
            \begin{equation}
                u(x) = v_0 + A x + h(x \cdot n) a.
            \end{equation}
        \end{enumerate}

        \item Suppose now we have a sequence $u_j \in W^{1,\infty}(\Omega;\bbR^m)$ converging weakly* to some $u$ in $W^{1,\infty}(\Omega;\bbR^m)$, and satisfying the \textit{approximate two-gradient inclusion} $d(\nabla u_j, \set{A,B}) \to 0$ in measure. If $\rank(A-B) \geq 2$, then $\nabla u_j \to A$ in measure, or $\nabla u_j \to B$ in measure.
    \end{enumerate}
\end{theorem}
\begin{proof}
    \textit{(ia)} Without loss of generality, $B = 0$, so we may write $\nabla u = A g$ for some $g \colon \Omega \to \bbR$. Assume further that $u \in C^\infty(\Omega;\bbR^m)$, so that $g \in C^\infty(\Omega)$. We will extend the result to $W^{1,\infty}$ functions by a mollification argument later. Now, since $\partial_i \partial_j u^k = \partial_j \partial_i u^k$ for all $i,j,k$, we have 
    \begin{equation}
        \tensor{A}{_j^k} \partial_i g = \tensor{A}{_i^k} \partial_j g.
    \end{equation}
    We claim that $\nabla g = 0$. Indeed, suppose otherwise, and let $x \in \Omega$ and $j = 1,\dots,d$ be such that $\partial_j g(x) \neq 0$. Then 
    \begin{equation}
        \tensor{A}{_i^k} = \frac{\tensor{A}{_j^k}}{\partial_j g(x)} \partial_i g(x) =: a^k \xi_i.
    \end{equation}
    But this means $A = a \xi$, showing that $A$ has rank at most 1. Contradiction. It follows by connectedness of $\Omega$ that $g$ is constant, finishing the proof of \textit{(ia)} in the smooth case.

    In the nonsmooth case, pick an open set $\Omega' \Subset \Omega$. For some $h > 0$ sufficiently small, we may mollify $u$ at level $h$ to obtain a function $u_h \in C^\infty(\Omega';\bbR^m)$ with $\nabla u_h = (\nabla u)_h = A g_h$ for some $g_h \in C^\infty(\Omega')$. So by our previous work, either $g_h = 0$ a.e., or $g_h = 1$ a.e. Now, since $g_h \to g$ a.e. in $\Omega'$, we must have $g = 0$ a.e. in $\Omega'$, or $g = 1$ a.e. in $\Omega'$. Since $\Omega' \Subset \Omega$ was arbitrary, we have finished \textit{(ia)}.

    \textit{(ib)} Assume without loss of generality that $A = 0$, so that $B = an^\T$. Choose a vector $v \in \bbR^d$ orthogonal to $n$.  Writing $\nabla u = B g$ as before, we have
    \begin{equation} \begin{aligned}
        \left. \odv{}{t} u(x + tv) \right\vert_{t=0} 
        &= \nabla u(x) v \\
        &= Bv g(x) \\
        &= a v \cdot n g(x) \\
        &= 0.
    \end{aligned} \end{equation}
    Thus $u$ is constant along any line orthogonal to $n$. In particular, $\nabla u$ and hence $g$ are also constant along any line orthogonal to $n$. Fix $x_0 \in \Omega$, and write $v_0 := u(x_0)$. Then, for any $x = x_0 + v + (x \cdot n) \in \Omega$ such that $v \in \bbR^d$ is orthogonal to $n$, we have 
    \begin{equation} \begin{aligned}
        u(x) - u(x_0)
        &= \int_0^1 \odv{}{t} u(x_0 + t(v + (x \cdot n)n)) \; \d{t} \\
        &= \int_0^1 \nabla u(x_0 + t(v + (x \cdot n)n))(v + (x \cdot n)n) \; \d{t} \\
        &= \parens{ \int_0^1 g(x_0 + t(v + (x \cdot n)n)) (x \cdot n) \; \d{t} } a \\
        &= h(x \cdot n) a.
    \end{aligned} \end{equation}
    By our previous observations, $h$ is independent of $v$. Furthermore, $h$ is Lipschitz (since it is defined by an integral), and $h'(x \cdot n) = g(x_0 + v + (x \cdot n)n) \in \set{0,1}$ a.e.

    \textit{(ii)} Again, assume without loss of generality that $B = 0$. Consider the sets 
    \begin{equation}
        D_j := \set{ x \in \Omega : \abs{\nabla u(x) - A} \leq \frac{\abs{A}}{2} }.
    \end{equation}
    By assumption, $\nabla u_j - A \bbone_{D_j} \to 0$ in measure. Since the sequence $\bbone_{D_j}$ is bounded in $L^\infty(\Omega)$, we can pass to a subsequence with $\bbone_{D_j} \weakstar \chi$ in $L^\infty(\Omega)$. Now, since $\nabla u_j - A \bbone_{D_j}$ is also a bounded sequence in $L^\infty(\Omega)$, we have, for any $w \in L^1(\Omega)$,
    \begin{equation} \begin{aligned}
        \int_\Omega (\nabla u_j - A \bbone_{D_j}) w \; \d{x} 
        &\leq \fhnorm{ \nabla u_j - A \bbone_{D_j} }_{L^\infty(\Omega)} \int_{\fhset{ \nabla u_j - A \bbone_{D_j} > \epsilon }} w \; \d{x} 
               + \int_\Omega \epsilon w \; \d{x} \\
        &\lesssim ( \fhabs{ \fhset{ \nabla u_j - A \bbone_{D_j} } } + \epsilon ) \norm{w}_{L^1(\Omega)} \\
        &\to \epsilon \norm{w}_{L^1(\Omega)}.
    \end{aligned} \end{equation}
    Therefore $\nabla u_j - A \bbone_{D_j} \weakstar 0$ in $L^\infty(\Omega)$. We conclude that $\nabla u_j \weakstar A \chi$ in $L^\infty(\Omega)$. 
    
    We will now show that $\chi = \bbone_D$ for some set $D \subseteq \Omega$. Since $A$ has rank at least 2, we can find a rank 2 minor $M$ with $M(A) \neq 0$. By weak* continuity of minors (lemma \ref{lem:minorWeakCty}), we have $M(\nabla u_j) \weakstar M(A) \chi^2$. On the other hand, since $\nabla u_j - A \bbone_{D_j} \weakstar 0$, we also have $M(\nabla u_j) \weakstar M(A) \chi$. Therefore $\chi^2 = \chi$, and so we can write $\chi = \bbone_D$ for some set $D \subseteq \Omega$. So $\nabla u = A \bbone_D$. By part \textit{(ia)}, we must have $\nabla u = A$ a.e., or $\nabla u = B = 0$ a.e. 

    It remains to prove $\nabla u_j \to \nabla u$ in measure. It suffices to prove this for the subsequence we have chosen, since we have assumed weak* convergence of the original sequence. Since $\bbone_{D_j} \weakstar \bbone_D$ in $L^\infty(\Omega)$, we clearly have $\fhnorm{\bbone_{D_j}}_{L^2(\Omega)} \to \fhnorm{\bbone_{D}}_{L^2(\Omega)}$. Similarly, $\bbone_{D_j} \weak \bbone_D$ in $L^2(\Omega)$. By the Radon-Riesz theorem, it follows that $\bbone_{D_j} \to \bbone_{D}$ in $L^2(\Omega)$ and hence in measure. So $\nabla u_j \to A \bbone_D = \nabla u$ in measure, as required.
\end{proof}

The Ball-James rigidity theorem allows us to answer the question from the beginning of the section. Namely, given two matrices $A,B \in \bbR^{m \times d}$ with $\rank(B - A) \geq 2$ and $\theta \in (0,1)$, is $\nu = \theta \delta_A + (1-\theta) \delta_B \in \bfY^\infty(\Omega;\bbR^{m \times d})$ a homogeneous gradient Young measure? The answer is no: suppose it were true, and let $u_j \in W^{1,\infty}(\Omega;\bbR^m)$ be a bounded sequence such that $\nabla u_j \weakstar \nabla u$ and $\nabla u_j \young \nu$ (up to subsequence). Then, since $\nu$ has support in $\set{A,B}$, lemma \ref{lem:ymConvergeInMeasure} implies $d(\nabla u_j, \set{A,B}) \to 0$ in measure. The Ball-James rigidity theorem implies $\nabla u = A$ a.e., or $\nabla u = B$ a.e. But the barycenter of $\nu$ is $\theta A + (1-\theta) B$, contradicting lemma \ref{lem:barycenterConvergence} which says $[\nu] = \nabla u$.


\chapter{Polyconvexity}
In our proof of weak lower semicontinuity for functionals with quasiconvex integrands in the last chapter, we relied on the $p$-growth bound $\abs{f(A)} \leq M(1 + \abs{A}^p)$. This is not a realistic physical assumption, since in elasticity theory, we should have $f(A) \to \infty$ as $\det{A} \to 0$. In this chapter, we will introduce a more restrictive notion of convexity which is more physically relevant.


\section{Polyconvex Functions}

A function $h \colon \bbR^{d \times d} \to \bbR \cup \set{\infty}$ is called \textit{polyconvex} if it can be written as a convex function of its minors. We will restrict attention to the notationally easier case $d = 3$, in which case $h$ is polyconvex if there exists a convex function $H \colon \bbR^{3 \times 3} \times \bbR^{3 \times 3} \times \bbR \to \bbR \cup \set{\infty}$ such that 
\begin{equation}
    h(A) = H(A,\cof{A},\det{A})
\end{equation}
for all $A \in \bbR^{3 \times 3}$. Certainly, convexity implies polyconvexity, but the determinant shows the converse is not true.
\begin{proposition}
    A polyconvex function is quasiconvex.
\end{proposition}
\begin{proof}
    Fix $A \in \bbR^{3 \times 3}$ and $w \in W_A^{1,\infty}(B(0,1);\bbR^3)$. Then by Jensen's inequality,
    \begin{equation} \begin{aligned}
        \dashint_{B(0,1)} h(\nabla w) \; \d x 
        &\geq H\parens{ \dashint_{B(0,1)} \nabla w \; \d x, \dashint_{B(0,1)} \cof{\nabla w} \; \d x, \dashint_{B(0,1)} \det{\nabla w} \; \d x } \\
        &= H(A,\cof{A},\det{A}) \\
        &= h(A),
    \end{aligned} \end{equation}
    where we use the fact that minors are null-Lagrangians.
\end{proof}


\section{Minimizers for Polyconvex Integral Functionals}

\begin{theorem}
    Let $\Omega \subseteq \bbR^3$ be a bounded Lipschitz domain. For $p \in (3,\infty)$, let $f \colon \Omega \times \bbR^{3 \times 3} \to \bbR \cup \set{\infty}$ be a Carath\'eodory integrand such that $f(x,\cdot)$ is polyconvex for a.e. $x \in \Omega$ and there exists $\mu > 0$ with $\mu\abs{A}^p \leq f(x,A)$ for a.e. $x \in \Omega$ and all $A \in \bbR^{3 \times 3}$. Let $q \in (1,\infty)$ be such that $\frac{1}{p} + \frac{1}{q} = 1$, and choose $b \in L^q(\Omega;\bbR^3)$ and $g \in W^{1-\frac{1}{p},p}(\partial\Omega;\bbR^3)$. Finally, assume there exists $u \in W^{1,p}_g(\Omega;\bbR^m)$ with $\det{\nabla u} > 0$ a.e. Then the variational problem
    \begin{equation} 
        \left\{ \begin{aligned}
            \text{minimize } & \scrF[u] := \int_\Omega f(x,\nabla u(x)) - b(x) \cdot u(x) \; \d{x} \quad u \in W^{1,p}(\Omega;\bbR^3) \\
            \text{subject to } & u = g \text{ on } \partial\Omega, \text{ and } \det{\nabla u} > 0 \text{ a.e.}
        \end{aligned} \right.
    \end{equation}
    has a solution.
\end{theorem}
\begin{proof}

\end{proof}

\chapter{Relaxation}
\input{s-Relaxation.tex}

\appendix
\chapter{Appendix}

\begin{theorem}[Friedrichs-Poincar\'e]
    Let $\Omega \subseteq \bbR^d$ be open and bounded. Fix $p \in [1,\infty]$, and suppose $q \in [1,\infty]$ satisfies the \textit{Friedrichs-Poincar\'e condition}
    \begin{equation}
        \begin{cases}
            q \in [1,p^*]    & \text{if } p < d, \\
            q \in [1,\infty) & \text{if } p = d, \\
            q \in [1,\infty] & \text{if } p > d,
        \end{cases}
    \end{equation}
    where $p^* = \frac{dp}{d - p}$ is the \textit{Sobolev conjugate} of $p$. Then there exists $C_P > 0$ such that
    \begin{equation}
        \norm{u}_{L^q} \leq C_P \norm{\nabla u}_{L^p}
    \end{equation}
    for all $u \in W_0^{1,p}(\Omega;\bbR^m)$.
\end{theorem}

\begin{theorem}[Rellich-Kondrachov] \label{thm:rellichKondrachov}
    Let $\Omega \subseteq \bbR^d$ be open and bounded. Fix $p \in [1,\infty]$, and suppose $u_j \weak u$ in $W^{1,p}(\Omega)$.
    \begin{enumerate}[label=(\roman*)]
        \item If $p < d$, then $u_j \rightarrow u$ in $L^q$ for all $q < p^*$.
        \item If $p = d$, then $u_j \rightarrow u$ in $L^q$ for all $q < \infty$.
        \item If $p > d$, then $u_j \rightarrow u$ uniformly.
    \end{enumerate}
\end{theorem}

\begin{lemma}[Mazur]
    Let $X$ be a Banach space. Suppose $x_j$ is a sequence in $X$ converging weakly to $x \in X$. Then there exists a sequence $y_j \in X$ of convex combinations of the form 
    \begin{equation}
        y_j = \sum_{k=j}^{N(j)} \theta^{(j)}_k x_k; \hspace{20pt} \theta^{(j)}_k \in [0,1]; \hspace{20pt} \sum_{k=j}^{N(j)} \theta^{(j)}_k = 1,
    \end{equation}
    such that $y_j \rightarrow x$ strongly in $X$.
\end{lemma}

\begin{lemma}[Pratt] \label{lem:pratt}
    Let $(X,\mu)$ be a measure space. Suppose $f_j,g_j \in L^1(X,\mu)$ are sequences converging pointwise a.e. to $f,g \in L^1(X,\mu)$ respectively, such that $\abs{f_j} \leq g_j$ for all $j$, and $\norm{g_j}_{L^1} \rightarrow \norm{g}$. Then 
    \begin{equation}
        \lim_{j \to \infty} \int_X f_j \; \d \mu = \int_X f \; \d \mu.
    \end{equation}
\end{lemma}
 
\end{document}

