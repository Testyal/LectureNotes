
Suppose $V_j \weak V$ in $L^2$. Given a Carathe\'eodory integrand $f$, we would like to know the limit of $\int_\Omega f(x,V_j(x)) \; \d x$. In general, this cannot be $\int_\Omega f(x,V(x)) \; \d x$. For example, [{\color{red} complete this}]

\section{The Fundamental Theorem}
Fix a bounded Lipschitz domain $\Omega \subseteq \bbR^d$. A family $\nu = (\nu)_{x \in \Omega}$ of probability measures on $\bbR^N$ is called \textit{weakly* measurable} if, for all $f \in C_0(\Omega \times \bbR^N)$, the function 
\begin{equation}
    x \mapsto \angles{f(x,\cdot), \nu_x} := \int_{\bbR^N} f(x,A) \; \d \nu_x(A)
\end{equation}
is Lebesgue measurable. To be able to go any further, we need to be able to extend measurability to more general functions.
\begin{proposition}
    Let $\nu = (\nu_x)_{x \in \Omega}$ be a weakly* measurable family of probability measures on $\bbR^N$. Then, for any Carath\'eodory integrand $f \colon \Omega \times \bbR^N \to \bbR$, the function 
    \begin{equation}
        x \mapsto \angles{f(x,\cdot),\nu_x}
    \end{equation}
    is Lebesgue measurable.
\end{proposition}
\begin{proof}
    {\color{red} maybe prove if you want}
\end{proof}
Because of how useful it is, we will be using this proposition in the rest of the chapter without warning.

For $p \in [1,\infty)$, we say $\nu$ is an $L^p$-\textit{Young measure} if it is weakly* measurable, and 
\begin{equation}
    \llangle \abs{\cdot}^p, \nu \rrangle := \int_\Omega \int_{\bbR^N} \abs{A}^p \; \d\nu_x(A) \, \d x < \infty.
\end{equation}
Similarly, we say $\nu$ is an $L^\infty$-\textit{Young measure} if it is weakly* measurable, and there exists a compact set $K \subseteq \bbR^N$ such that $\supp{\nu_x} \subseteq K$ for a.e. $x \in \Omega$. For $p \in [1,\infty]$, the set of all such $L^p$-Young measures is denoted $\bfY^p(\Omega;\bbR^N)$.

A subset $A \subseteq \bfY^p(\Omega;\bbR^N)$ is said to be \textit{bounded} if 
\begin{equation}
    \sup_{\nu \in A} \aangles{\abs{\cdot}^p,\nu} < \infty \quad \text{for } p < \infty,
\end{equation}
and for $p = \infty$, there exists a compact set $K \subseteq \bbR^N$ such that for all $\nu \in A$, $\supp{\nu_x} \subseteq K$ for a.e. $x \in \Omega$.

The most important theorem in Young measure theory is the following:
\begin{theorem}[Fundamental Theorem of Young Measure Theory] \label{thm:fundamentalTheoremOfYoungMeasures}
    For $p \in [1,\infty]$, let $V_j \in L^p(\Omega;\bbR^N)$ be a bounded sequence. Then there exists a subsequence of $V_j$, and a Young measure $\nu \in \mathbf{Y}^p(\Omega;\bbR^N)$ such that
    \begin{equation} \label{eq:youngLimitForContinuousFunctions}
        \lim_{j \to \infty} \int_\Omega f(x,V_j(x)) \; \d x = \int_\Omega \int_{\bbR^N} f(x,A) \; \d\nu_x(A) \, \d x
    \end{equation}
    for all $f \in C_0(\Omega \times \bbR^N)$.
\end{theorem}
We call $\nu$ the Young measure \textit{generated by} $V_j$, and we write $V_j \young \nu$. Our goal will be to prove the fundamental theorem by showing a more general compactness result for Young measures.

Given a sequence $\nu_j$ of weakly* measurable families of probability measures, we say $\nu_j$ \textit{converges weakly*} to another weakly* measurable family $\nu$ if $\fhaangles{f,\nu_j} \rightarrow \fhaangles{f,\nu}$ for all $f \in C_0(\Omega \times \bbR^N)$. As usual, we write $\nu_j \weakstar \nu$. 
The proof of theorem \ref{thm:fundamentalTheoremOfYoungMeasures} is an immediate consequence of the following Young measure compactness and approximation lemmas.
\begin{lemma}[Compactness] \label{lem:youngMeasureCompactness}
    Let $p \in [1,\infty]$, and let $\nu^{(j)} \in \bfY^p(\Omega;\bbR^N)$ be a bounded sequence of Young measures. Then there is a subsequence of $\nu^{(j)}$ and a weakly* measurable family $\nu = (\nu_x)_{x \in \Omega}$ such that $\nu^{(j)} \weakstar \nu$. 
\end{lemma}
\begin{lemma}[Approximation] \label{lem:youngMeasureApproximation}
    Let $\nu^{(j)} \in \bfY^p(\Omega;\bbR^N)$ be a bounded sequence of Young measures with $\nu^{(j)} \weakstar \nu$. Let $f \colon \Omega \times \bbR^N \to \bbR$ be a Carath\'eodory integrand satisfying
    \begin{enumerate}[label={\rm (\roman*)}]
        \item (Uniform integrability)
        \begin{equation}
            \sup_{j \in \bbN} \int_\Omega \abs{\fhangles{f(x,\cdot),\nu^{(j)}_x}} \; \d x < \infty,
        \end{equation}
        
        \item (Equiintegrability)
        \begin{equation}
            \lim_{h \to \infty} \sup_{j \in \bbN} \fhaangles{ \abs{f} \bbone_{ \{(x,A) \in \Omega \times \bbR^N : \abs{f(x,A) \geq h}\} }, \nu^{(j)} } = 0.
        \end{equation}
    \end{enumerate}
    Then 
    \begin{equation}
        \lim_{j \to \infty} \fhaangles{f,\nu^{(j)}} = \fhaangles{f,\nu}.
    \end{equation}
\end{lemma}
\begin{corollary} \label{cor:lowerSemicontinuityOfYoungMeasures}
    For $p \in [1,\infty]$, let $\nu_j \in \bfY^p(\Omega;\bbR^N)$ be a bounded sequence converging weakly* to $\nu$. Let $f \colon \Omega \times \bbR^N \to \bbR$ be a Carath\'eodory integrand. Then the lower semicontinuity estimate 
    \begin{equation}
        \aangles{f,\nu} \leq \liminf_{j \to \infty} \aangles{f,\nu^{(j)}}
    \end{equation}
    holds. In particular, $\nu$ is an $L^p$-Young measure.
\end{corollary}
\begin{proof}
    For $h > 0$, define $f_h(x,A) := \min\{f(x,A),h\}$. Then (i) and (ii) of the approximation lemma hold, so
    \begin{equation}
        \liminf_{j \to \infty} \aangles{f,\nu^{(j)}} \geq \lim_{j \to \infty} \aangles{f_h,\nu^{(j)}}
                                                               = \aangles{f_h,\nu}.
    \end{equation}
    By the monotone convergence theorem, $\aangles{f_h,\nu} \rightarrow \aangles{f,\nu}$ as $h \to \infty$. The final statement follows by considering $f(x,A) = \abs{A}^p$ for $p < \infty$, and for $p = \infty$ doing some cool shit {\color{red} do this cool shit}
\end{proof}

The proof of the fundamental theorem is now immediate:
\begin{proof}[Proof of theorem \ref{thm:fundamentalTheoremOfYoungMeasures}]
    Define the \textit{elementary Young measure} $\delta[V_j] \in \bfY^P(\Omega;\bbR^N)$ by $\delta[V_j]_x := \delta_{V_j(x)}$. Then 
    \begin{equation}
        \fhaangles{ f,\delta[V_j] } = \int_\Omega f(x,V_j(x)) \; \d x
    \end{equation}
    for all $f \in C_0(\Omega \times \bbR^N)$. It is easy to see that $\nu^{(j)} := \delta[V_j]$ if a bounded sequence in $\bfY^p(\Omega;\bbR^N)$. Lemma \ref{lem:youngMeasureCompactness} and corollary \ref{cor:lowerSemicontinuityOfYoungMeasures} then complete the proof.
\end{proof}

\begin{proof}[Proof of lemma \ref{lem:youngMeasureApproximation}]
    {\color{red} prove using scorza-dragoni if u want :)}
\end{proof}
\begin{proof}[Proof of lemma \ref{lem:youngMeasureCompactness}] \renewcommand{\qedsymbol}{}
    Define measures $\mu^{(j)} \in \scrM(\Omega \times \bbR^N)$ by 
    \begin{equation}
        \angles{f,\mu} = \int_\Omega \int_{\bbR^N} f(x,A) \; \d\nu^{(j)}_x(A) \, \d x
    \end{equation}
    for $f \in C_0(\Omega \times \bbR^N)$. This defines a measure since $\scrM(\Omega \times \bbR^N) \cong C_0(\Omega \times \bbR^N)^*$. Note that 
    \begin{equation}
        \abs{\fhangles{f,\mu^{(j)}}} \leq \abs{\Omega} \norm{f}_{L^\infty}
    \end{equation}
    for all $f \in C_0(\Omega \times \bbR^N)$. The sequence $\mu^{(j)}$ is therefore bounded in $C_0(\Omega \times \bbR^N)^*$, which, by the Banach-Alaoglu theorem, implies the existence of a subsequence and a measure $\mu \in \scrM(\Omega \times \bbR^N)$ such that $\mu^{(j)} \weakstar \mu$. To continue, we will need the following disintegration theorem.
\end{proof}
\begin{theorem}[Disintegration of Measures]
    Let $\Omega \subset \bbR^d$ be open and $\mu \in \scrM(\Omega \times \bbR^N)$ a Radon measure. Then there exists a weakly* measurable family $\nu = (\nu)_{x \in \Omega}$ of probability measures such that 
    \begin{equation}
        \angles{f,\mu} = \int_\Omega \int_{\bbR^N} f(x,A) \; \d\nu_x(A) \, \d\kappa(x)
    \end{equation}
    for all $f \in C_0(\Omega \times \bbR^N)$, where $\kappa \in \scrM(\Omega)$ is given by $\kappa(B) = \mu(B \times \bbR^N)$ for $B \subseteq \Omega$ a Borel set.
\end{theorem}
\begin{proof}[Proof of lemma continued]
    Using the disintegration theorem, let $\nu$ be the weakly* measurable family of probability measures corresponding to $\mu$. We need to show $\kappa = \scrL^d \restrict \Omega$. Suppose $p < \infty$, and let $U \subseteq \Omega$ be open. Then by lower semicontinuity of weak* convergence of measures, we have 
    \begin{equation}
        \kappa(U) = \mu(U \times \bbR^N) \leq \liminf_{j \to \infty} \mu^{(j)}(U \times \bbR^N) = \abs{U}.
    \end{equation}
    On the other hand, for $K \subseteq \Omega$ compact, upper semicontinuity gives us 
    \begin{equation} \begin{aligned}
        \mu(K \times \overline(B(0,R))) &\geq \limsup_{j \to \infty} \mu^{(j)}(K \times \overline{B(0,R)}) \\
                                        &=    \int_K \int_{\overline{B(0,R)}} \; \d\nu^{(j)}_x \, \d x \\
                                        &\geq \int_K \int_{\overline{B(0,R)}} 1 - \frac{\abs{A}^p}{R^p} \; \d\nu^{(j)}_x(A) \, \d x \\
                                        &\geq \abs{K} - \frac{1}{R^p} \sup_{j \in \bbR^N} \aangles{ \abs{\cdot}^p,\nu^{(j)} } \\
    \end{aligned} \end{equation}
    Taking $R \to \infty$ on both sides and using continuity of $\mu$, we conclude $\kappa(K) \geq \abs{K}$. Since $\kappa$ is a Radon measure, it follows that $\kappa = \scrL^d \restrict \Omega$.
\end{proof}

Let's consider some examples. Note that to prove $V_j \young \nu$, it suffices to show 
\begin{equation}
    \lim_{j \to \infty} \int_\Omega \phi(x) h(V_j(x)) \; \d x = \int_\Omega \phi(x) \angles{h,\nu_x} \; \d x
\end{equation}
for all $\phi \in C_0(\Omega)$ and $h \in C_0(\bbR^N)$. Indeed, functions of the form $(x,A) \mapsto \phi(x)h(A)$ are dense in $C_0(\Omega \times \bbR^N)$.

A Young measure $\nu = (\nu_x)_{x \in \Omega}$ is \textit{homogeneous} if $x \mapsto \nu_x$ is constant a.e.

\begin{example}
    \begin{enumerate}[label=(\arabic*)]
        \item On $\Omega = (0,1)$, define $u := \bbone_{(0,\frac{1}{2})} - \bbone_{(\frac{1}{2},1)}$, and extend this to $\bbR$ periodically. Define $u_j(x) := u(jx)$ for $x \in \Omega$. We claim $u_j$ generates the homogeneous $L^\infty$-Young measure $\nu = \frac{1}{2}(\delta_{+1} + \delta_{-1}) \in \bfY^\infty(\Omega)$. Inded, fix $\phi \in C_0(\Omega)$ and $h \in C_0(\bbR^N)$. Then, since $h$ is bounded,
        \begin{equation} \begin{aligned}
            \int_\Omega \phi(x) h(u_j(x)) \; \d x &= \sum_{k=0}^{j-1} \int_{k/j}^{(k+1)/j} \phi(x) h(u_j(x)) \; \d x \\
                                                &= \sum_{k=0}^{j-1} \int_{k/j}^{(k+1)/j} \phi\parens{\frac{k}{j}} h(u_j(x)) \; \d x + \frac{1}{j} O\parens{\omega\parens{\frac{1}{j}}} \\
                                                &= \sum_{k=0}^{j-1} \frac{1}{j} \phi\parens{\frac{k}{j}} \int_0^1 h(u(y)) \; \d y + \frac{1}{j} O\parens{\omega\parens{\frac{1}{j}}} \\
                                                &= \sum_{k=0}^{j-1} \frac{1}{j} \phi\parens{\frac{k}{j}} \int_\bbR h \; \d\nu + \frac{1}{j} O\parens{\omega\parens{\frac{1}{j}}} \\
                                                &\to \int_\Omega \phi(x) \; \d x \int_\bbR h(A) \; \d\nu(A) \\
                                                &= \int_\Omega \int_{\bbR} \phi(x) h(A) \; \d\nu_x(A) \, \d x,
        \end{aligned} \end{equation}
        as required. In the third line, we made the change of variables 
        \begin{equation}
            x = \frac{k}{j} + \frac{1}{j}y,
        \end{equation}
        and we also used the fact that $\phi$ has modulus of continuity $\omega$.

        \item Let $\Omega \subseteq \bbR^2$ be a bounded Lipschitz domain, and let $A,B \in \bbR^{2 \times 2}$ be \textit{rank-one connected}, in the sense that $B-A$ has rank at most 1. We may then write $B - A = an^\T$ for some $a,n \in \bbR^2$. Let $\theta_A, \theta_B \in (0,1)$ be such that $\theta_A + \theta_B = 1$. For $x \in \bbR^2$, define 
        \begin{equation}
            u(x) := Ax + \parens{ \int_0^{x \cdot n} \bbone_{{k \in \bbZ} [k,k+\theta_B))}(t) \; \d t } a.
        \end{equation}
        Then, for $j \in \bbN$, define $u_j \in W^{1,\infty}(\Omega;\bbR^2)$ by $u_j(x) := j^{-1} u(jx)$. Then 
        \begin{equation}
            \nabla u_j(x) = A + \bbone_{{k \in \bbZ} [k,k+\theta_B))}(jx \cdot n) an^\T.
        \end{equation}
        We now claim $\nabla u_j$ generates the homogeneous Young measure $\nu = \theta_A \delta_A + \theta_B \delta_B \in \bfY^\infty(\Omega;\bbR^{2 \times 2})$.
    \end{enumerate}
\end{example}

Suppose a bounded sequence $V_j \in L^p(\Omega;\bbR^N)$ generates the Young measure $\nu \in \bfY^p(\Omega;\bbR^N)$. Given a Carath\'eodory integrand $f \colon \Omega \times \bbR^N \to \bbR$, the approximation lemma tells us that if 
\begin{enumerate}[label=(\roman*)]
    \item (Uniform integrability)
    \begin{equation}
        \sup_{j \in \bbN} \int_\Omega \abs{f(x,V_j(x))} \; \d x < \infty,
    \end{equation}
    and 
    \item (Equiintegrability)
    \begin{equation}
        \lim_{h \to \infty} \sup_{j \in \bbN} \int_{\set{\abs{f(\cdot,V_j)} \geq h}} \abs{f(x,V_j(x))} \bbone_{\abs{f} \geq h}(x,A) \; \d x = 0,
    \end{equation}
\end{enumerate}
then
\begin{equation}
    \lim_{j \to \infty} \int_\Omega f(x,V_j(x)) \; \d x = \int_\Omega \int_{\bbR^N} f(x,A) \; \d\nu_x(A) \, \d x.
\end{equation}
This property along with corollary \ref{cor:lowerSemicontinuityOfYoungMeasures} will be important in the rest of these notes. Also note that this convergence works fine for vector-valued integrands $f$, by considering each component separately.

\section{Young Measures and Convergence}
Given a Young measure $\nu \in \bfY^p(\Omega;\bbR^N)$, its \textit{barycenter} $[\nu] \in L^p(\Omega;\bbR^N)$ is defined by 
\begin{equation}
    [\nu](x) = [\nu_x] := \int_\Omega A \; \d\nu_x(A).
\end{equation}

\begin{lemma} \label{lem:barycenterConvergence}
    For $p \in (1,\infty]$, let $V_j \in L^p(\Omega;\bbR^N)$ be a bounded sequence generating a Young measure $\nu \in \bfY^p(\Omega;\bbR^N)$. Then $V_j \weak [\nu]$ if $p < \infty$, and $V_j \weakstar [\nu]$ if $p = \infty$.
\end{lemma}
\begin{proof}
    First, we take $p = \infty$. Take a test function $\phi \in L^1(\Omega)$. Since $V_j \in L^\infty(\Omega;\bbR^N)$ is bounded, the function $\phi V_j$ is clearly uniformly integrable and equiintegrable. We can then take $f(x,A) := A$ to see 
    \begin{equation}
        \lim_{j \to \infty} \int_\Omega \phi(x) V_j(x) \; \d x = \int_\Omega \phi(x) [\nu](x) \; \d x.
    \end{equation}
    It follows that $V_j \weakstar [\nu]$ in $L^\infty$.

    Now take $p < \infty$. By Banach-Alaoglu, $V_j$ is weakly precompact in $L^p$, so by Dunford-Pettis, {\color{red} finish}
\end{proof}
Note this lemma fails for $p = 1$, a counterexample is the concentrating sequence $V_j = j\bbone_{(0,\frac{1}{j})}$.

The following lemma demonstrates a useful technique we can use to prove convergence in measure for a number of functions, so long as we have a Young measure at hand.
\begin{lemma} \label{lem:ymConvergeInMeas}
    Let $\nu \in \bfY^p(\Omega;\bbR^N)$, $p \in [1,\infty]$ be generated by a bounded sequence $V_j \in L^p(\Omega;\bbR^N)$. Then 
    \begin{enumerate}[label={\rm (\roman*)}]
        \item For any compact set $K \subseteq \bbR^N$, we have $d(V_j,K) \to 0$ in measure if and only if $\supp{\nu_x} \subseteq K$ for a.e. $x \in \Omega$.
        \item For any $V \in L^p(\Omega;\bbR^N)$, we have $V_j \to V$ in measure if and only if $\nu_x = \delta_{V(x)}$ for a.e. $x \in \Omega$.
    \end{enumerate}
\end{lemma}
\begin{proof}
    Let $f \colon \Omega \times \bbR^N \to [0,1]$ be a Carath\'eodory integrand. Fix $\epsilon \in (0,1)$. Since $f$ is bounded, $f(\cdot,V_j)$ is uniformly integrable and equiintegrable, so by the approximation lemma and the Markov inequality, 
    \begin{equation} \begin{aligned}
        \limsup_{j \to \infty} \abs{\set{ x \in \Omega : f(x,V_j(x)) \geq \epsilon }} &\leq \lim_{j \to \infty} \frac{1}{\epsilon} \int_\Omega f(x,V_j(x)) \; \d x \\
                                                                                      &= \frac{1}{\epsilon} \int_\Omega \int_{\bbR^N} f(x,A) \; \d\nu_x(A) \, \d x,
    \end{aligned} \end{equation}
    so if $\angles{f(x,\cdot),\nu_x} = 0$ for a.e. $x \in \Omega$, then $f(\cdot,V_j) \to 0$ in measure. Conversely, we also have 
    \begin{equation} \begin{aligned}
        \int_\Omega \int_{\bbR^N} f(x,A) \; \d\nu_x(A) \, \d x &= \lim_{j \to \infty} \int_\Omega f(x,V_j(x)) \; \d x \\
                                                               &\leq \epsilon\abs{\Omega} + \abs{\set{x \in \Omega : f(x,V_j(x)) \geq \epsilon}}.
    \end{aligned} \end{equation}
    So if $f(\cdot,V_j) \to 0$ in measure, then $\angles{f(x,\cdot),\nu_x} = 0$ for a.e. $x \in \Omega$.

    To finish the proof, it remains to select appropriate integrands $f$. For (i), take 
    \begin{equation}
        f(x,A) := \frac{d(A,K)}{1 + d(A,K)}.
    \end{equation}
    Then $d(V_j,K) \to 0$ in measure if and only if $f(\cdot,V_j) \to 0$ in measure, which, by the above work, is equivalent to 
    \begin{equation}
        0 = \angles{f(x,\cdot),\nu_x} = \int_{\bbR^N} \frac{d(A,K)}{1 + d(A,K)} \; \d\nu_x(A) \quad \text{for a.e. } x \in \Omega.
    \end{equation}
    This is equivalent to $\supp{\nu_x} \subseteq K$ for a.e. $x \in \Omega$.

    For (ii), take 
    \begin{equation}
        f(x,A) := \frac{\abs{A - V(x)}}{1 + \abs{A - V(x)}}.
    \end{equation}
    Then $V_j \to V$ in measure if and only if $f(\cdot,V_j) \to 0$ in measure, which is therefore equivalent to 
    \begin{equation}
        0 = \angles{f(x,\cdot),\nu_x} = \int_{\bbR^N} \frac{\abs{A - V(x)}}{1 + \abs{A - V(x)}} \; \d\nu_x(A) \quad \text{for a.e. } x \in \Omega.
    \end{equation}
    This is equivalent to $\nu_x = \delta_{V(x)}$ for a.e. $x \in \Omega$.
\end{proof}

\section{Gradient Young Measures}
Let $\nu \in \bfY^p(\Omega;\bbR^{m \times d})$ be a Young measure. If there exists a bounded sequence $u_j \in W^{1,p}(\Omega;\bbR^m)$ such that $\nabla u_j$ generates $\nu$, then we say $\nu$ is a \textit{gradient Young measure}, and write $\nu \in \bfGY^p(\Omega;\bbR^{m \times d})$.
\begin{lemma} \label{lem:gradientYM}
    For $p \in (1,\infty]$, let $\nu \in \bfGY^p(\Omega;\bbR^{m \times d})$, and let $u \in W^{1,p}(\Omega;\bbR^m)$ be an \textit{underlying deformation} of $u$, which is to say $[\nu] = \nabla u$. Then there exists a bounded sequence $u_j \in W^{1,p}(\Omega;\bbR^m)$ such that $\supp(u_j - u) \Subset \Omega$ and $\nabla u_j$ generates $\nu$. Furthermore, if $p < \infty$, then $\abs{\nabla u_j}^p$ is equiintegrable.
\end{lemma}

Homogeneous gradient Young measures satisfy a certain averaging property, characterized by the following lemmas.
\begin{lemma}
    For $p \in [1,\infty]$, let $\nu \in \bfGY^p(\Omega;\bbR^{m \times d})$ have underlying deformation $u \in W^{1,p}(\Omega;\bbR^m)$ such that $u$ has linear boundary values. That is, $u \vert_{\partial\Omega}(x) = Ax$ for all $x \in \partial\Omega$ and some linear map $A \colon \bbR^d \to \bbR^m$. Then, for any bounded Lipschitz domain $D \subseteq \bbR^d$, there exists a homogeneous gradient Young measure $\overline{\nu} \in \bfGY^p(D;\bbR^{m \times d})$ such that 
    \begin{equation}
        \int_{\bbR^{m \times d}} h(A) \; \d\overline{\nu}(A) = \dashint_\Omega \int_{\bbR^{m \times d}} h(A) \; \d\nu_x(A) \, \d x.
    \end{equation}
    for any continuous $h \colon \bbR^{m \times d} \to \bbR$ with $p$-growth.
\end{lemma}
An important special case:
\begin{lemma}[Riemann-Lebesgue]
    For $p \in [1,\infty]$, let $u \in W^{1,p}(\Omega;\bbR^m)$ have linear boundary values. Then there is a homogeneous gradient Young measure $\overline{\delta[\nabla u]} \in \bfGY^p(\Omega;\bbR^{m \times d})$ such that 
    \begin{equation}
        \int_{\bbR^{m \times d}} h(A) \; \d\overline{\delta[\nabla u]} = \dashint_\Omega h(\nabla u(x)) \; \d x
    \end{equation}
    for all continuous $h \colon \bbR^{m \times d} \to \bbR$ with $p$-growth.
\end{lemma}